\documentclass[a4paper]{article} 
\usepackage[dvipdfmx]{color,graphicx}
%\usepackage{nruby} %$B%k%S(B
%\usepackage{shadow}  % $B1FIU$N;M3QOH(B
%\usepackage{fancybox} % $B$$$m$$$m6E$C$?OH(B
%\usepackage{longtable} %  $B%Z!<%8$r8Y$0Bg$-$JI=(B( Tabular$B$b$"$k(B )
%\usepackage{mathrsfs}  %  $B2VJ8;z%m!<%^;zB>$K$b$$$m$$$m$"$k(B
%\usepackage{amsmath,amsthm,amssymb} % $B%"%a%j%+?t3X2q$N5-9f$J$I(B
%\usepackage{ascmac}  %  $B3Q$N4]$$OH$GJ#?t9T$r0O$`(B
%\usepackage{multicolpar} % $BBPLuJ8$N$h$&$K:81&$KJ8$rJ,$1$F=q$/(B
%\usepackage{fancyvrb} % $B=q$$$F$"$k$^$^$N=PNO$KBP$9$k<~JUAu>~(B
%\usepackage{alltt}  % verbatim$B$G$O:$$k$H$-$K;H$($k(B?
%\usepackage{here}
%\setlength{\textwidth}{15.5cm}  % $B%F%-%9%H$NI}$r;XDj(B
%\setlength{\textheight}{24cm}  % $B%F%-%9%H$N9b$5$r;XDj(B
\title{Brems energy sampling}
\author{KK}
\begin{document}
 \maketitle
 For  treating electron bremshtralung, we need to get the total brems 
 cross-section  (say, probability or brems occurence / r.l)  and must be able
 to sample the emitted gamma ray energy. Normally, the total cross-section
 diverges when the minimum gamma ray energy  is set to 0 (though, if the
  LPM is effective, divergnece dose not happen).

  The gamma ray energy is  usually expressed by $k=E_\gamma/E_e$ where
  $E_e$ is the electron totalc energy.  To avoid divergence, we normally
  set a minimum value of  $k=k_{m}$ above which  gamma ray energy is
  sampled. This is applied also for the LPM case.

  However, some times we may want to use a  lower $k_{m}$ than
  predefined one. To prepare a sampling table for each different $k_m$
  is not practical.

  

  \section{Sampling Method}
  We express the differential brems cross-section by
  \begin{equation}
     \frac{d\sigma}{dk}=f(k)
  \end{equation}
  This is dependent on $E_e$.
  For a typical $k_m$, we make a table with various $E_e$'s.
First,  the total cross-section, i.e,  integration from $k=k_{m} $ to  $1-m_e/Ee$,
\begin{equation}
   T(k_m)= \int_{k_m} f(k)dk
\end{equation}
is  tabulated. 


  For $k\sim 1$, $f(k)$ is not a simple function, so we must
   also make a sampling table for $k$,
  beforehand.
  For a lower $k_c$ ($<k_m,$),
  $T(k_c)$ can be obtained by a simple formula, since $f(k)$
  becomes a simple function for sufficiently small $k$.
  That is,
  when the LPM is to be applied:
  \begin{equation}
 f(k) =f_1\equiv  f_m\cdot(k/k_m)^{-1/2} =f_{m1}/\sqrt{k}  
  \end{equation}
  where $f_m=f(k_m)$ and $f_{m1}=f_m\sqrt{k_m}$.
When the LPM should not be applied:
  \begin{equation}  
f(k)=f_2\equiv  f_m\cdot(k/k_m)^{-1} = f_{m2}/k
  \end{equation}
  where  $f_{m2}=f_mk_m$.
  Both of $f_{m1}$ and $f_{m2}$ are tabulated together with
  $T(k_m)$
  \begin{equation}  
T(k_c)= \int_{k_c}  f(k)dk = \int_{k_c}^{k_m}f(k)dk + T(k_m)
\end{equation}
Defining
  \begin{equation}  
   t_i(k_c) \equiv  \int_{k_c}^{k_m}f_idk
=  \left \{
\begin{array}{l}
2f_{m1}(\sqrt{k_m}-\sqrt{k_c}) \hspace{1cm}(i=1)\\
f_{m2}\log(k_m/k_c)\hspace{1.8cm}(i=2)
\end{array}
\right.
  \end{equation}
%
We can express
  \begin{equation}  
T(k_c)= t_i(k_c) + T(k_m)
\end{equation}
 For a given uniform random number $0<u<1$,
if $u>t_i(k_c)/T(k_c)$,  we may sample $k$ by using the sampling table.
Otherwize, we may  sample $k$ from $f_i$.  Using a new $u$,
  \begin{equation}  
   k=  \left \{
\begin{array}{l}
k_c(1+u(\sqrt{k_m/k_c} -1))^2\hspace{1cm}(i=1)\\
k_c(\frac{k_m}{k_c})^u\hspace{3.5cm}(i=2)\\
\end{array}
\right.
  \end{equation}
  
  \section{Summary}
  \begin{itemize}
   \item Besides $T(k_m)$, $f_{m1}$ and $f_{m2}$ must be tabulated as
	 a function of  $E_e$.
   \item If  $f_{m1}$ or $f_{m2}$ is not given in the table, old style
	 format is
	 assumed, and giving an new $k_c$ is not allowed (negelcted)
   \item  If a new $k_c$ is larger than the current $k_m$,  it is
	  neglected
	  (or $k_c=k_m$ is forced. )
	 
   \item Normally the minimum gamma ray energy must be $\ge$ 1 keV.
	 (It can be smaller without error;
	 If $E_e<$100 MeV,
	  the absolute
	 $E_\gamma$ minimuum has been  set to be 1 keV in default).
	 At higher energies, $k_m=10^{-5}$ is default and this is
	 enough small for many of applications.  If $E_e$ is as
	 high as 10 TeV, it means the minimum gamma energy becomes
	 100 MeV. One may think this is too high, but for many
	 applications where energy deposit is measured,  results
	 with this cut would give a result which is almost the
	 same one by the default setup
   	 
   \item The total cross-section table, say, BrTXL(mxBrTXL),
	 is now made to be BrTXL(2, mxBrTXL).  The first index is
	 for the total cross-section, and the second one is
	 for $f_{m1}$ or $f_{m2}$.

	 BrTXL, BrTXH, BrTXS, BrTXS2 are also changed.
	 
  \end{itemize}
\end{document}

