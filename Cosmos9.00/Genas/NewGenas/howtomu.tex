\documentclass[a4j]{jarticle}
\usepackage[dvips]{color,graphicx}
\makeatletter
\def\neareq{\mathrel{\mathpalette\neareqx@align{\smash.}}}
\def\neareqx@align#1#2{\lower.2ex\vbox{\baselineskip\z@skip\lineskip\z@
    \def\finsm@sh{\ht\z@.2ex \dp\z@.2ex\box\z@}%
    \ialign{$\m@th#1\hfil##\hfil$\crcr#2\crcr=\crcr#2\crcr}}}
\makeatother

\def\dfrac#1#2{{\displaystyle\frac{#1}{#2}}}
        \makeatletter
        \def\le{\mathrel{\mathpalette\gl@align<}}
        \def\ge{\mathrel{\mathpalette\gl@align>}}
        \def\gl@align#1#2{\lower.6ex\vbox{\baselineskip\z@skip\lineskip\z@
                \ialign{$\m@th#1\hfil##\hfil$\crcr#2\crcr=\crcr}}}
        \makeatother
%  \def\thefootnote{\fnsymbol{footnote}}
%\pagestyle{myheadings}  
%\pagestyle{empty}
% \thispagestyle{empty}
% \markright{K.K  \today}

\begin{document}
\section{Systematics for muons}


\subsection{Goal}
For a given primary type, energy $E_0$, zenith angle($\cos \theta$), 
  observation depth $z$, core distance $R$,
  we can know  the muon number density, muon erengy spectrum and arrival time
distrituion whihc is a function of enery and cordistance of each muon.

\subsection{First step}
\begin{enumerate}
\item  Air shower size, $N_e$, is not a good parameter to deal with
  muons.  The primary energy $E_0$ and center of gravity (COG) of the
  AS transtion curve are better parameters.

\item The average number of muons, $N_\mu$, as a function of depth
  (transition curve of muons) is well expressed  by the following
  function.
\end{enumerate}
\begin{equation}
 N_\mu(x) = a x^m  \exp(-b x^n)
\end{equation}

The following coefficients are only to show some rough guide and
must be  updated by using a number of M.C data.
We must see the heavy primary case separately (say, use effective
primary energy, or simple super-position model).
(We express depth in g/cm$^2$,  energy TeV.
\begin{eqnarray}
  x &= & z/z_m \\
  z_m &= & 66.7\log_{10}(E_0)+625 \\
  a &= &N_\mu \exp(b)\\
  N_\mu &= & 29 E_0^{0.8} \cos\theta^{0.61} \\
  b & = & 3.7 \\
  m &= & bn \\
  n &= &0.606E_0^{0.0417}
\end{eqnarray}

The forms  will need some correction if $\cos\theta<0.5$ (i.e, $>60^\circ$).
(For example, at $\cos\theta \sim 0.5$, $ b=9.5,
 z_m= ( 66.7\log_{10}(E_0/TeV) + 600) *0.55,
    n  =0.606E_0^{0.0417} * 0.55$ 
are better parameters?)

The formulas could be used for $z>150$. However, normally we don't need
to see shallow depths; if we concentrate only at $z>500$ or so, 
it would be easier to construct accurate formula  valid upto $\cos\theta < 0.5$.


With these formulas, we can fix the average muon number, $<N_\mu>$.
We have to introduce fluctualtion. We note that the distribution of
$N_\mu/<N_\mu>$ is well approximated by a
Gaussian.  Howerver, sampling $N_\mu$ from  this Gaussian distribution
has no explicit correlation with air shower development.  We, therefore,
see a correlation between $N_\mu/<N_\mu>$ and $COG$. 
We define  $<N_mu^n> = < N_\mu/<N_\mu> >$, then, 
 at $z>400$,
\begin{equation}
          <N_\mu^n> = (COG/ A)^B
\end{equation}
where  
As noted above, $N_\mu/<N_\mu> >$
is distributed like Gaussian with the mean of $<N_\mu^n>$ and
$\sigma\sim 0.3$.
  $<N_\mu^n> $ runs from 0.7 to 1.5.
$A$ and $B$ are weakly dependent on $\cos\theta, E_0 and z$ 
with typical values of $ A\sim 320, B=0.22\sim  0.35 $. 
We must parameterize $A, B, \sigma$. 

After these are established, we can proceed as follows:
\begin{enumerate}
\item For a given primary type, $E_0, \cos\theta, z$, we can fix $<N_\mu>$
\item Then,  $<N_\mu^n>$ is obtained from $COG, A, B$.
\item From $\sigma$ and $<N_\mu^n>$, we can sample $N_\mu/<N_\mu>$.
\end{enumerate}


\subsection{Second step}
There are almost no muons below 50MeV (kinetic energy). If we can neglect
energy depencnce of energy deposit in the plastic scintillator,
we can neglect energy spectrum of muons. 


\subsubsection{Lateral speread}
The lateral speread $R_\mu$.  Its average $<R_\mu>$ and $COG$
have correlation and it is roughly expressed by
\begin{equation}
  <R_\mu> \sim  \exp(-COG/B)
\end{equation}
with $B= 700\sim 790$ ( at deeper depths,  large value~
The distribuion of $x=R_\mu/<R_\mu>$ is approximated by
\begin{equation}
 A x^n \exp(-bx^m)
\end{equation}
$A$ is determined by the normalization. The rough values of the parameters
are $n = 0.4\sim 2,    b\sim 3,   m\sim 0.5$, but must be
investigated dependence of other parameters.
However, at $x > 0.1$, we get almost universal values. Therefore
we may investigae the region of $x<01.$ separately (which is
about  $R_\mu<30$ m and not so important).

\subsubsection{Arrival time, $T_\mu$}

At least we need correlation with $R_\mu$.  If we can neglect the energy
dependence of energy deposit in the detector, we don't need 
energy spectrum.


Let's denote the geometrical average by $<T>_g$. Then,  
\begin{equation}
 <T>_g = A R_\mu^B  
\end{equation}
 ($R_\mu$ in m).  A rough estimation is
$ A\sim  0.01$ and  $B \sim 1.7$. Both should depend on $COG$.


It is not easy to find a good function to epress $x=T_\mu/<T_\mu>$.
One possible way would be to use $y=log_{10}(x)$ and to
express it by two gaussian functions, $G~_1(y)+G_2(y)$,
where $G_i(y)=A_i \exp(-( (y-<y>)/\sigma_i)^2/2))$.
$<y>$ is common to $G_1 and G_2$.
  $A_i=A_i(R_\mu)$ and $\sigma_i=\sigma_i(R_\mu)$.  The $R_\mu$
dependece is large.  A rough formulat is

\begin{equation}
<y> \sim 0.5 x \log_{10}(R_\mu) - 1.4
\end{equation}
As   $R_mu= 10 \rightarrow  *1000$, 
\begin{eqnarray}
A_1& =& 0.6 \rightarrow  1.4\\
\sigma_1 & = & 0.9 \rightarrow 0.3 \\
A_2 & = & 1.4 \rightarrow 1.2  \\
\sigma_2 & =& 0.9\rightarrow 0.3 \\
\end{eqnarray}

\end{document}







 






