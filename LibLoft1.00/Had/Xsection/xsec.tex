\documentclass[a4paper]{jarticle} 
\usepackage[dvipdfmx]{color,graphicx}
%\usepackage{nruby} %$B%k%S(B
%\usepackage{shadow}  % $B1FIU$N;M3QOH(B
%\usepackage{fancybox} % $B$$$m$$$m6E$C$?OH(B
%\usepackage{longtable} %  $B%Z!<%8$r8Y$0Bg$-$JI=(B( Tabular$B$b$"$k(B )
%\usepackage{mathrsfs}  %  $B2VJ8;z%m!<%^;zB>$K$b$$$m$$$m$"$k(B
%\usepackage{amsmath,amsthm,amssymb} % $B%"%a%j%+?t3X2q$N5-9f$J$I(B
%\usepackage{ascmac}  %  $B3Q$N4]$$OH$GJ#?t9T$r0O$`(B
%\usepackage{multicolpar} % $BBPLuJ8$N$h$&$K:81&$KJ8$rJ,$1$F=q$/(B
%\usepackage{fancyvrb} % $B=q$$$F$"$k$^$^$N=PNO$KBP$9$k<~JUAu>~(B
%\usepackage{alltt}  % verbatim$B$G$O:$$k$H$-$K;H$($k(B?
%\usepackage{here}
%\setlength{\textwidth}{15.5cm}  % $B%F%-%9%H$NI}$r;XDj(B
%\setlength{\textheight}{24cm}  % $B%F%-%9%H$N9b$5$r;XDj(B
\title{Xsection treatment}
\author{The Author}
\begin{document}
 \maketitle
 \begin{itemize}
  \item  Consider a medium consisting of  $n-$elements.  E.g, PWO:
	   Pb + W + 4O.  $n=3$.  
  \item The $i-$th element is denoted by the suffix.
	  An example is shown for PWO.   
	
	\begin{description}
	 \item[$A_i$]:  Mass number of $i-$th element.
		    element(i).A. For i=1,3, the value is    (207.2, 183.92, 16).
		    The component is denoted by using ``.'',
		    though  formal notation should use ``\%', like
		    element(i)\%A.
	 	 \item[$Z_i$]:  charge of the $i-$th element.
	 element(i).Z)	    (82,74,8)
	 \item[$N_i$]:  number of $i-$th element.
If we follow the notation above, this should be
		    element(i).No. But we use  No(i) like media\%No(i).
		    (1,1,4) .    This is however,
		    normalize to (1/6, 1/6, 2/3).  The original number
		    is kept as .OrigNo(:)
	 \item[$\sigma_i$]:  cross-section of  the $i-$th element,
		    at a given energy and for a given projectile.  So this
		    changes for each collision.
                    For the PWO case,  $\sigma_1$ is the cross-section
		    of
		    Pb, $\sigma_2$  of W, $\sigma_3$ of O (not  of  4O).

		    These are not kept as variables.
		    
	 \item[nsigma(i)]    $No(i) \sigma_i $.  These
		    are
		    used to sample the target among elements in a
		    medium.
	 \item[media.xs ] $\sum $nsigma(:).  This is used to
		    sample the  MFP of the hadronic  collision  in the medium. 

	\end{description}
	

    	
 \end{itemize}

\end{document}
