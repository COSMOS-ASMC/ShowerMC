%%%%%%%%%%%%%%%%%%%%%%%%%%%%%%%%%%%%%%%%%%%%%%%%%%%%%%%%%%%%%%%%%%%%%%%%%%%%%%%%
%Manual for JAM                                                      % % last
%modified Apr. 22 2000                                                   % %
%%%%%%%%%%%%%%%%%%%%%%%%%%%%%%%%%%%%%%%%%%%%%%%%%%%%%%%%%%%%%%%%%%%%%%%%%%%%%%%%
\documentclass[]{article}
%define page size: these two lines are for A4 paper
%\setlength{\textheight}{245mm} \setlength{\topmargin}{-5mm}

%define page size: these two lines are for US legal size paper 
%to enable them, remove % at the beginning of the lines
\setlength{\textheight}{240mm}
\setlength{\topmargin}{10mm}

%define page size: common for A4 and US legal
\setlength{\headheight}{0mm}
\setlength{\headsep}{-15mm}
\setlength{\footskip}{12mm}
\setlength{\textwidth}{160mm}
\setlength{\oddsidemargin}{0mm}
\setlength{\evensidemargin}{0mm}
\setlength{\textwidth}{180mm}
\setlength{\oddsidemargin}{-10mm}
\setlength{\evensidemargin}{-20mm}

\def\VERSION{1.009}

%line spacing
\renewcommand{\baselinestretch}{1.0}

%new list environments to replace itemize and enumerate
\newenvironment{Itemize}{\begin{list}{$\bullet$}%
{\setlength{\topsep}{0.2mm}\setlength{\partopsep}{0.2mm}%
\setlength{\itemsep}{0.2mm}\setlength{\parsep}{0.2mm}}}%
{\end{list}}
\newcounter{enumct}
\newenvironment{Enumerate}{\begin{list}{\arabic{enumct}.}%
{\usecounter{enumct}\setlength{\topsep}{0.2mm}%
\setlength{\partopsep}{0.2mm}\setlength{\itemsep}{0.2mm}%
\setlength{\parsep}{0.2mm}}}{\end{list}}

%list environment suitable for manuals
\newenvironment{entry}%
{\begin{list}{}{\setlength{\topsep}{0mm} \setlength{\itemsep}{0mm}
\setlength{\parskip}{0mm} \setlength{\parsep}{0mm}
\setlength{\leftmargin}{20mm} \setlength{\rightmargin}{0mm}
\setlength{\labelwidth}{18mm} \setlength{\labelsep}{2mm}}}%
{\end{list}}
\newenvironment{subentry}%
{\begin{list}{}{\setlength{\topsep}{0mm} \setlength{\itemsep}{0mm}
\setlength{\parskip}{0mm} \setlength{\parsep}{0mm}
\setlength{\leftmargin}{10mm} \setlength{\rightmargin}{0mm}
\setlength{\labelwidth}{18mm} \setlength{\labelsep}{2mm}}}%
{\end{list}}

\newcommand{\ttt}[1]{{\tt#1}}
\newcommand{\itemt}[1]{\item[{\tt #1}\hfill]}
\newcommand{\iteme}[1]{\item[{\tt #1}]}
%\newcommand{\itemt}[1]{\item[\texttt{#1}\hfill]}

\newcommand{\comment}[1]{}
\newcommand{\bold}[1]{\mbox{\boldmath $#1$}}    %       bold symbol
\newcommand{\rr}{{\bold{r}}}
\newcommand{\pp}{{\bold{p}}}
\newcommand{\RR}{{\bold{R}}}
\newcommand{\PP}{{\bold{P}}}
\newcommand{\srt}{\mbox{$\sqrt{s}$}}
\newcommand{\plab}{\mbox{$p_{\rm lab}$}}
\newcommand{\Elab}{\mbox{$E_{\rm lab}$}}
\newcommand{\Kbar}{\mbox{${\bar K}$}}
\newcommand{\sigtot}{\mbox{$\sigma_{tot}(s)$}}
\newcommand{\sigel}{\mbox{$\sigma_{el}(s)$}}
\newcommand{\sigtR}{\mbox{$\sigma_{\mbox{t-R}}(s)$}}
\newcommand{\sigtS}{\mbox{$\sigma_{\mbox{t-S}}(s)$}}
\newcommand{\sigbw}{\mbox{$\sigma_{BW}(s)$}}
\newcommand{\sigsS}{\mbox{$\sigma_{\mbox{s-S}}(s)$}}
\newcommand{\sigpiY}{\mbox{$\sigma_{\pi Y}(s)$}}

\newcommand{\sigdiff}{\mbox{$\sigma_{{\rm diff}}(s)$}}
\newcommand{\signondiff}{\mbox{$\sigma_{{\rm nondiff}}(s)$}}
\newcommand{\sigsoft}{\mbox{$\sigma_{{\rm soft}}(s)$}}
\newcommand{\sigjet}{\mbox{$\sigma_{{\rm jet}}(s)$}}
\newcommand{\sigann}{\mbox{$\sigma_{{\rm ann}}(s)$}}
\newcommand{\sigch}{\mbox{$\sigma_{{\rm ch}}(s)$}}
\newcommand{\Bbar}{\mbox{$\bar B$}}

\newcommand{\Dl}{\mbox{$\Delta(1232)$}}
\newcommand{\Ds}{\mbox{$\Delta^*$}}
\newcommand{\Ns}{\mbox{$N^*$}}
\newcommand{\Mx}{\mbox{${\cal M}$}}                    % matrix element

%roman names for particles in math mode
\def\GeV{\mbox{\rm GeV}}
\def\MeV{\mbox{\rm MeV}}
\def\fm{\mbox{\rm fm}}

\renewcommand{\a}{\rm{a}}
\renewcommand{\b}{\rm{b}}
\renewcommand{\c}{\rm{c}}
\renewcommand{\d}{\rm{d}}
\newcommand{\e}{\rm{e}}
\newcommand{\f}{\rm{f}}
\newcommand{\g}{\rm{g}}
\newcommand{\hrm}{\rm{h}}
\newcommand{\lrm}{\rm{l}}
\newcommand{\n}{\rm{n}}
\newcommand{\p}{\rm{p}}
\newcommand{\q}{\rm{q}}
\newcommand{\s}{\rm{s}}
\renewcommand{\t}{\rm{t}}
\renewcommand{\u}{\rm{u}}
\newcommand{\A}{\rm{A}}
\newcommand{\B}{\rm{B}}
\newcommand{\D}{\rm{D}}
\newcommand{\F}{\rm{F}}
\renewcommand{\H}{\rm{H}}
\newcommand{\J}{\rm{J}}
\newcommand{\K}{\rm{K}}
\renewcommand{\L}{\rm{L}}
\newcommand{\Q}{\rm{Q}}
\newcommand{\R}{\rm{R}}
\newcommand{\T}{\rm{T}}
\newcommand{\W}{\rm{W}}
\newcommand{\Z}{\rm{Z}}
\newcommand{\bbar}{\overline{\rm{b}}}
\newcommand{\cbar}{\overline{\rm{c}}}
\newcommand{\dbar}{\overline{\rm{d}}}
\newcommand{\fbar}{\overline{\rm{f}}}
\newcommand{\pbar}{\overline{\rm{p}}}
\newcommand{\qbar}{\overline{\rm{q}}}
\newcommand{\sbar}{\overline{\rm{s}}}
\newcommand{\tbar}{\overline{\rm{t}}}
\newcommand{\ubar}{\overline{\rm{u}}}
\newcommand{\Fbar}{\overline{\rm{F}}}
\newcommand{\Qbar}{\overline{\rm{Q}}}

\newcommand{\lsim}{\mbox{\raisebox{-0.6ex}{$\stackrel{<}{\sim}$}}\:}
\def\GeVc{{\rm GeV/c}}




% A useful Journal macro
\def\Journal#1#2#3#4{{#1} {\bf #2}, #3 (#4)}

% Some useful journal names
\def\NCA{\em Nuovo Cimento}
\def\NIM{\em Nucl. Instrum. Methods}
\def\NIMA{{\em Nucl. Instrum. Methods} A}
\def\NP{\em Nucl. Phys.}
\def\NPA{{\em Nucl. Phys.} A}
\def\NPB{{\em Nucl. Phys.} B}
\def\PLB{{\em Phys. Lett.}  B}
\def\PRL{\em Phys. Rev. Lett.}
\def\PR{{\em Phys. Rev.}}
\def\PRD{{\em Phys. Rev.} D}
\def\PRC{{\em Phys. Rev.} C}
\def\ZPA{{\em Z. Phys.} A}
\def\ZPC{{\em Z. Phys.} C}
\def\PR{{\em Phys. Rep.}}
\def\CPC{{\em Comp. Phys. Comm.}}
\def\MPLA{{\em Mod. Phys. Lett.} A}
\def\ARPS{{\em Annu. Rev. Part. Sci.}}
\def\PTPsuppl{{\em Prog. Theor. Phys. Suppl.}}
\def\RMP{{\em Rev.Mod.Phys.}}
\def\ANP{{\em Annals Phys.}}
\def\PPNP{{\em Prog. Part. Nucl. Phys.}}
\def\PTP{{\em Prog. Theor. Phys.}}
\def\JP{{\em J. Phys.}}

\def\etal{{\it et al.}}

%fraction of page devoted to figures and other floats
\setcounter{topnumber}{1}
\setcounter{bottomnumber}{1}
\renewcommand{\textfraction}{0.1}
\renewcommand{\topfraction}{0.9}
\renewcommand{\bottomfraction}{0.9}
\renewcommand{\floatpagefraction}{0.8}

%one caption and figure/table, indented but spanning whole page
\newlength{\captivewidth}
\setlength{\captivewidth}{\textwidth}
\addtolength{\captivewidth}{-10mm}
\newcommand{\captive}[1]{\rule{5mm}{0mm}%
\begin{minipage}{\captivewidth}%
\caption[small]{#1}\end{minipage}}

%minimal separation between rows in table 
\newlength{\tablinsep}
\setlength{\tablinsep}{0.73\baselineskip}

%size of column (half page width + blank in middle)
\newlength{\halfpagewid}
\setlength{\halfpagewid}{0.5\textwidth}
\addtolength{\halfpagewid}{-10mm}
 
%set level for table of contents
\setcounter{tocdepth}{2}



\begin{document}

\sloppy
%\begin{flushright}
%  version 1.005
%\end{flushright}


\title{\bf JAM \VERSION:
A microscopic simulation program
 for ultra-relativistic nuclear collisions
%\footnote{}
}
\author{Yasushi Nara \thanks{E-mail:ynara@hadron31.tokai.jaeri.go.jp}\\
  {\it Advanced Science Research Center, 
        Japan Atomic Energy Research Institute}, \\
 {\it Tokai, Naka, Ibaraki 319-1195, Japan}
     and\\
 {\it Physics Department,
  Brookhaven National Laboratory, Upton, N.Y. 11973, U.S.A.}}

\date{}
\maketitle

\begin{abstract}
\noindent
JAM is a simulation program which
 is designed to simulate (ultra-) relativistic nuclear collisions
 from initial stage of nuclear collision to final state interaction
 in hadronic gas stage
 and
 can be used from low($\Elab=1-10$AGeV) energy.
In JAM, hadrons and their excited states
% and on-shell partons
% which are produced from hard parton-parton scattering
 are explicitly propagated in space-time by the cascade method.
In order to describe nuclear collisions consistently from low to high energy,
  elementary hadron-hadron collisions are modeled by resonance production
  at low energies, string excitations at soft region and
  hard parton-parton scattering at collider energies according to
  the important physics at each energy range.
\end{abstract}

PACS: 25.75.G, 14.20.G, 24.85.+p

%\newpage

\section{Program Summary}

{\noindent\em Title of program:} JAM 1.0\\

{\noindent\em Program obtainable from:}
 CPC Program Library, Queen's University of Belfast, N. Ireland
 and from the author at ynara@hadron03.tokai.jaeri.go.jp\\


{\noindent\em Computer for which the program is designed:}
 Pentium, Dec alpha, Hewlett Packard UX-A-9000  SPARC stations
and other computers with a FORTRAN 77 compiler.\\


{\noindent\em Computers on which it has been tested:}
  DEC Workstation,  SUN workstation, Pentium personal computer.\\

{\noindent\em Operating systems under which the program has been tested:}
OSF1 V4.0 878 alpha, Free-BSD, Linux, Sun-OS 5.5.1.\\

{\noindent\em Programming language used:} FORTRAN 77\\

{\noindent\em Memory required to execute with typical data:} 1,450 words\\ 

{\noindent\em No.of bits in a word:} 64\\

{\noindent\em Number of lines in distributed program:} 64253\\

{\noindent\em Keywords:} ultrarelativistic heavy ion collisions,
cascade, hadron, parton, pQCD, Boltzmann equation. \\


{\noindent\em Nature of physical problem}

\noindent
The microscopic description of ultra-relativistic nuclear collisions
is necessary to understand the time evolution of 
hot and dense matter produced in heavy-ion collisions
 from non-equilibrium stage.
A model which can simulate consistently 
 from low to high incident energy nuclear collisions
 are required for the
systematic studies of the bombarding energy dependence on some observable
and the treatment of the final state interactions among produced particle.

\bigskip

{\noindent\em Method of solution}

\noindent
 The space-time evolution of hadrons including 
 excited states
is solved by the cascade method.
Low energy total and elastic hadron-hadron cross sections
 are parameterized to experimental data if possible.
Inelastic hadron-hadron collisions are modeled by the resonance
picture at low energy,
string picture at intermediate energy
and hard parton-parton scattering at high energy.

\bigskip

{\noindent\em Typical running time}

\noindent
 The running time depends on the bombarding energy,system size and
 impact-parameter.
For example, on the pentium II 266MHz on Free-BSD 2.2.6R:

Si(14.6AGeV/c)+Al b=1.5fm 1m 33s per 100events on DEC alpha.


{\noindent\em Unusual features of the program}\\
\noindent none

%\newpage

\bigskip
\begin{flushleft}
{\large \bf LONG WRITE-UP}\\
\end{flushleft}


\section{Introduction}

In ultra-relativistic heavy ion collisions,
 reaction dynamics is generally complex.
We have to solve basically many-body system involved several
  hundred particles at initial stage
  but due to the large number of multi-particle productions,
  number of particles are reached to about the several (hundred) thousands
 depending on both the incident energy and impact parameter.
We may use Monte-Carlo simulation to solve these many-body system.
In order to investigate and understand its dynamics,
  event generators have been made such as
 FRITIOF~\cite{fritiof}, LUCIAE~\cite{luciae},
 DPM~\cite{dpm}, VENUS~\cite{venus},
 HIJING~\cite{hijing}, HIJET~\cite{hijet}.
These models do not follow explicitly space-time trajectory but
assume Glauber geometry for the treatment of AA collisions.
Microscopic hadronic transport models such as
  RQMD~\cite{rqmd1,rqmd2}, QGSM~\cite{qgsm}, ARC~\cite{arc},
  ART~\cite{art}, UrQMD~\cite{urqmd} and HSD~\cite{hsd}
  have been used to describe the nucleus-nucleus collisions
%  especially at AGS($\sim 10$AGeV) and SPS($\sim 200$AGeV) energies
  from first $NN$ collision stage
  to the final state interaction among produced particles.
These microscopic transport models follow the space-time trajectory
  of the particles and can extract detailed space-time information
  on the reaction dynamics in nuclear collisions.

Boltzmann-Uehling-Uhlenbeck (BUU)~\cite{buu} approach
  based on the classical kinetic one-body transport equation
  including the Pauli-blocking of the final state of the colliding nucleons
  is the first attempt to describe heavy ion collisions
  at the energy range 30AMeV $\sim$ 2GeV,
  by microscopic transport model
  in which two-body collision term are realized
  by the intra-nuclear cascade model~\cite{inc1,inc2,inc3}
  and nuclear meanfield is solved by the test-particle method~\cite{buu}.
Quantum molecular dynamics (QMD) model~\cite{qmd}
  has been developed in order to simulate the time evolution of the
  n-body phase space distribution function.
Both relativistic version of BUU and QMD have been also
  appeared which are called RBUU~\cite{rbuu} and RQMD~\cite{rqmd1} respectively.
ART and HSD are an extension of BUU and RBUU model to the high energies,
  while RQMD and UrQMD are based on the spirit of QMD model.
The most important element for high energy nuclear collisions \Elab$>10$AGeV
  in above mentioned transport models are incoherent superposition of
  two-body collisions among particles.
%In the present version of JAM1.0,
%  any mean field effect
%  is not included,
%  therefore, the applicability of JAM1.0 is restricted to only high energies.
Some observable may strongly depend on the mean field,
  for example, the mean field effect on the transverse flow at AGS energies
  ~\cite{flow,art}.



In ultra-relativistic heavy ion collisions at collider energies,
 hard and semihard processes are expected to be important~\cite{hijing}.
Multiple minijet production with initial and final state radiation
  has been included into event generator for nuclear collision
  in HIJING~\cite{hijing} and VNI~\cite{vni} based on perturbative QCD (pQCD).
In HIJING, the number of minijets per inelastic $pp$ collision
  is calculated using an eikonal formalism and AA collision
  are realized by the Glauber type multiple $NN$ collisions.
VNI is a Monte Carlo implementation of the parton cascade model
  in which space-time evolution of on-shell as well as off-shell partons
  are followed by cascade method.
Any multiple parton-parton collisions are, therefore,
  automatically included in VNI.
Interaction among the produced partons would be important.
Parton cascade models VNI~\cite{vni} and ZPC~\cite{zpc}
 are designed to implement theses parton dynamics to study
 space-time evolution of partons.

JAM has been developed based on the above mentioned models
 for the concept that
 it should reduce hadronic transport model like BUU and QMD at low energies,
 in order to describe consistently nuclear collision from low to 
 high energy, in addition to be able to treat the final state
 interaction among produced hadrons,
 while at collider energies, multiple minijets production is 
 included.
% and space-time evolution of the partons from hard scattering
% can be followed.
Although, in present version of JAM, these requirements has not been
 completely implemented,
The main features included in JAM\VERSION~ are as follows.
 (1) At low energy, inelastic hadron-hadron ($hh$) collisions are modeled by
     the resonance productions based on the idea from RQMD and UrQMD.
 (2) Nuclear mean field is simulated based on the BUU theory.
 (3) Above resonance region, soft string excitation is implemented
     along the lines of the HIJING model~\cite{hijing}.
 (4) Multiple minijet production is also included in the same way
     as the HIJING model in which jet cross section and the number
     of jet is calculated using an eikonal formalism for
     perturbative QCD (pQCD) and hard parton-parton scatterings
     with initial and final state radiation are simulated
     using PYTHIA~\cite{pythia} program.


%In principle,
% all parameters of the microscopic model for nuclear collisions
% should be determined and fixed from elementary $pp$ or $ep$ data.
%However, there are some free parameters like formation time
%  which can be not determined by the basic theory at present.






\section{Description of the model}

In this section, we will present the main aspect of JAM~\cite{jam}.

\subsection{Main features of the model}

For the description of AA collisions in  cascade model,
 the trajectories of all hadrons as well as resonances
 including produced particles are followed explicitly
   as a function of space and time.
The main components of our model are as follows.
(1)  Nuclear collision is assumed to be described by the
   sum of independent binary $hh$ 
   (possibility parton-parton, parton-hadron)
   collisions.
   $hh$ collisions are realized by the closest distance approach.
  Two particles will collide if their distance is smaller than
  $\sqrt{\sigma(s)/\pi}$, here $\sigma(s)$ represents the total cross
  section at the c.m. energy $\srt$.
  As a default option, any mean field is not included, therefore 
  the trajectory of hadrons are straight line in between two-body
  collisions or decays.
  As a results of $hh$ collision, hadrons are assumed to be
  excited to either resonance or string at soft collisions,
  in the case of hard scattering, on-shell partons and remnants.
(2) The initial positions of each nucleons are sampled by the
    parameterized distribution of nuclear density.
   Fermi motion of nucleons are assigned
   according to the local Fermi momentum.
(3) all established hadronic states (see appendix~\ref{app:particle})
    are explicitly included
    with explicit iso-spin states as well as their anti-particles
   and they can propagate in space-time.
(4) the inelastic $hh$ collisions produce resonances at low energies
   while at high energies
   ( $\sim 4\GeV$ in $BB$ collisions $\sim 3\GeV$ in $MB$ collisions
    and $\sim 2\GeV$ in $MM$ collisions)
  color strings are formed and they decay into hadrons according to the
  Lund string model with some formation time.
 Formation point and time are determined
  by assuming yo-yo formation point~\cite{bialas}.
 This gives roughly formation time of 1fm/c
  with string tension $\kappa=1$GeV/fm.
(5)
Hadrons which have original constituent quarks can scatter with hadrons
  assuming the additive quark cross section within a formation time.
The importance of this quark(diquark)-hadron interaction
 for the description of baryon stopping
at CERN/SPS energies was reported by Frankfurt group~\cite{rqmd1,urqmd}.
(6) Pauli-blocking for the final nucleons in two-body collisions
    are also considered.
(7)
 Low energy baryon-baryon, baryon-meson and  meson-meson rescattering
 are also included assuming resonance/string excitation picture
 in order to treat final state interaction of hadronic gas.
Extending the particle table of PYTHIA, baryon and meson resonances
 are explicitly propagated and they can rescatter.




\subsection{Modeling hh collisions}

In order to consistently describe nucleus-nucleus collisions
for the large energy regime from $\sim 1\GeV$ to collider energies, 
hadron-hadron collisions are basically modeled by the resonance production at
 $\srt<4\GeV$,
soft string excitation at $4<\srt<50\GeV$,
minijet production at $\srt>10\GeV$.
Total hadronic cross section is divided by
\begin{equation}
  \sigma_{tot}(s)=\sigel+\sigtR+\sigbw+\sigtS+\sigsS+\sigann+\sigch ~,
\end{equation}
where $\sigel$~ is elastic,
      $\sigtR$~ t-resonance production,
      $\sigbw$~ s-channel resonance formation,
      $\sigsS$~ s-channel string formation,
      $\sigann$~ annihilation of $\Bbar B$,
      $\sigch$~ charge exchange cross sections respectively.
$\sigtS$~ contains the soft string excitation and hard parton-parton scattering
which can be written within the eikonal formalism for pQCD~\cite{hijing}
\begin{equation}
\sigtS=2\pi\int^{\infty}_0 db^2\left[1-e^{-\chi(b,s)}\right],
\end{equation}
where $\chi(b,s)$ is defined as
\begin{equation}
  \chi(b,s)=\left({1\over2}\sigsoft+{1\over2}\sigjet\right) T_N(b,s).
\end{equation}
with overlap function $T_N(b,s)$
   and soft \sigsoft\ and hard \sigjet\ cross sections.
The \sigsoft~ is usually decomposed into diffractive part and
  low $p_T$ non-diffractive part.
The \sigjet\ rrepresents the cross section of hard parton scattering
  and can be calculated by perturbative QCD (pQCD) as
  \begin{equation}
    \sigjet  =  \int \int \int d x_1 \, d x_2 \, d \hat{t} \,
     f_1(x_1, Q^2) \, f_2(x_2, Q^2) \, \frac{d \hat{\sigma}}{d \hat{t}}.
  \end{equation}
  with parton distribution function $f(x_i,Q^2)$ and $Q$ is
  a typical momentum transfer in the partonic process.
The $d \hat{\sigma}/d \hat{t}$ expresses the differential
cross section for a partonic scattering.
PYTHIA is used to generate hard scattering including initial- and
final-state radiation.
The space-time points of partons produced by the hard scattering 
 are calculated applying uncertainty principle~\cite{eskola1}.


Particle productions at low energies($\srt\leq 4.0$ for the BB collisions)
are modeled via the excitation and decay of resonances.
Inelastic cross sections are assumed to be filled up with
the resonance formations \sigtR up to $E_{cm}=$3-4GeV.
The gap between experimental inelastic cross section
 and resonance formation cross sections
 are assigned to be string formation.
The following resonance excitations are implemented in NN
 collisions
\begin{itemize}
\item[(1)] $NN \to  N\Delta(1232)$,
  \ \ (2) $NN \to  NN^* $,
  \ \ (3) $NN \to  \Delta(1232)\Delta(1232)$,
\item[(4)] $NN \to  N\Delta^* $,
  \ \ (5) $NN \to  N^*\Delta(1232) $,
\ \ (6) $NN \to  \Delta(1232)\Delta^* $,
\ \ (7) $NN \to  N^*N^*  $,
\item[(8)] $NN \to  N^*\Delta^* $,
\ \ (9) $NN \to  \Delta^*\Delta^* $.
\end{itemize}
The strength of each branches are determine from the exclusive
pion production data~\cite{CernHera}.


The cross section for the production of resonances may be written by
\begin{equation}
 \label{eq:cross}
 {d\sigma_{12\to34}\over d\Omega}=
    {(2S_3+1)(2S_4+1)\over 64\pi^2 s p_{12}}
    \int\int p_{34} |\Mx|^2 A(m_3^2)A(m_4^2)dm_3^2dm_4^2
\end{equation}
where $S_i, i=3,4$ express the spin of the particles in the final state.
Mass distribution function $A(m_i^2)$ for nucleons is just a $\delta$-function,
while that for resonances is given
by the relativistic Breit-Wigner function
\begin{equation}
  A(m^2)={1\over \cal N}{m_R\Gamma(m)\over (m^2-m_R^2)^2 + m_R^2\Gamma(m)^2}.
 \label{eq:rlorentz}
\end{equation}
where $\cal N$ denotes the normalization constant.
 In this paper,
we use simply take ${\cal N} = \pi$ which is a value in the case of
a constant width.
The full width $\Gamma(m)$ is a sum of all partial decay width
$\Gamma_R(MB)$ for resonance $R$ into mesons $M$ and baryons $B$
which depends on the momentum of the decaying particle~\cite{rqmd2,urqmd}:
\begin{equation}
 \Gamma_R(MB)=\Gamma^0_R(MB) {m_R\over m}
              \left({p_{cms}(m)\over p_{cms}(m_R)}\right)^{2\ell+1}
   {1.2 \over 1+0.2\left({p_{cms}(m)\over p_{cms}(m_R)}\right)^{2\ell+1}}
   \label{eq:width}
\end{equation}
where $\ell$ and $p_{cms}(m)$ are
 the relative angular momentum
 and 
 the relative momentum in the exit channel in their rest frame.
%
Neglecting the mass dependence of the matrix element,
   masses of resonances are generated according to the distribution
   \begin{equation}
      P(m_3,m_4)= 4m_3m_4 p_{34}A(m_3^2)A(m_4^2)dm_3dm_4
   \end{equation}
%
%
Note that we do not give the cross sections for each resonance production
  but for the average cross section.
We need to know the matrix element $|\Mx|^2$ for all resonances
  from the formula Eq.(\ref{eq:cross}).
In Ref.~\cite{Teis}, the matrix elements are fitted to reproduce
the pion production cross sections up to two-pion productions
 as well as $\eta$ production cross section assuming the constant value
 for the matrix element.
In our model, we assume simply that
 each resonance production cross section can be selected
 according to the probability:
\begin{equation}
  \label{eq:resprob}
  P(R_i,R_j) \sim (2S_i+1)(2S_j+1)\int\int
                    p_{ij}(m_i,m_j)A_i(m_i)A_j(m_j)dm_idm_j ,
\end{equation}
where $A_i$ represents the Breit-Wigner function.
This simple assumption gives reasonable results for the pp data.


\subsection{Inverse processes}

In the processes for resonance absorption,
  we use a generalized detailed balance formula~\cite{Wolf2,detbal1,detbal2}
  which takes the finite width of the resonance mass into account.
The differential cross section for the reaction $(3,4)\to(1,2)$
can be expressed by the cross section for $(1,2)\to(3,4)$;
\begin{equation}
  {d\sigma_{34\to 12}\over d\Omega}=
       {(2S_1+1)(2S_2+1)\over (2S_3+1)(2S_4+1)}
          {p^2_{12} \over p_{34}}
    {d\sigma_{12\to 34}\over d\Omega}
     {1\over \int\int p_{34} A(m_3)A(m_4)d(m_3^2)d(m_4^2)}~.
\end{equation}


%%%%%%%%%%%%%%%%%%%%%%%%%%%%%%%%%%%%%%%%%%%%%%%%%%%%%%%%%%%%%%%%%
\subsection{Meson-Baryon, Meson-Meson Collisions}
\label{subsec:MBMM}
%%%%%%%%%%%%%%%%%%%%%%%%%%%%%%%%%%%%%%%%%%%%%%%%%%%%%%%%%%%%%%%%%

We now turn to the explanation of meson-baryon ($MB$) and meson-meson
($MM$) collisions.
We also use
 resonance/string excitation model for $MB$ and $MM$ collisions.

Total cross section for $\pi N$ incoming channel is assumed to be decomposed to
\begin{equation}
 \sigtot^{\pi N}=\sigbw+\sigel +\sigsS+\sigtS,
\end{equation}
%
where \sigel, \sigbw, \sigsS, and \sigtS~ denote
the $t$-channel elastic cross section,
the $s$-channel resonance formation cross section
	with the Breit-Wigner form, 
the $s$-channel and $t$-channel string formation cross sections, respectively.
We neglect the $t$-channel resonance formation cross section at a
  energy range of 
%$<\sqrt{s}\sim 2$GeV.
$\sqrt{s}\lsim 2$GeV.
%
The $t$-channel elastic cross section \sigel~ was determined so that
the sum of the elastic component of
 the $s$-channel Breit-Wigner cross section \sigbw~
 and $t$-channel elastic cross section \sigel~ 
reproduces the experimental elastic data for $\pi N$ interaction.
%
Above the $\Delta(1232)$ region, $t$-channel elastic cross section
becomes non-zero in our parameterization.
%
String formation cross sections (\sigsS~ and \sigtS) are 
 determined to fill up the difference between
 experimental total cross section and $\sigbw+\sigel$.
%
We calculate the resonance formation cross section \sigbw\ 
using the Breit-Wigner formula~\cite{Brown,rqmd1}
(neglecting the interference between resonances),
\begin{equation}
    \label{eq:bw}
\sigma(MB \to R)= {\pi(\hbar c)^2 \over p_{cm}^2}
         \sum_{R}  |C(MB,R)|^2  {(2S_R+1) \over (2S_M+1)(2S_B+1)}
                     {\Gamma_R(MB)\Gamma_R(tot) \over
                      (\sqrt{s}-m_R)^2+\Gamma_R(tot)^2/4} \ .
\end{equation}
The momentum dependent decay width
Eq.~(\ref{eq:width}) are also used for the calculation of
 decay width in Eq.~(\ref{eq:bw}).
$S_R$, $S_B$ and $S_M$ denote the spin of
 the resonance, the decaying baryon and meson respectively.
The sum runs over resonances,
   $R=N(1440)\sim N(1990)$ and $\Delta(1232)\sim\Delta(1950)$.
Actual values for these parameters are 
 taken from the Particle Data Group~\cite{PDG96}
 and adjusted
 within an experimental error bar
to get reasonable fit for $MB$ cross sections.


%%%%%%%%%%%%%%%%%%%%%%%%%%%%%%%%%%%%%%%%%%%%%%%%%%%%%
% aK-N total and elastic
%\FIGxakn
%%%%%%%%%%%%%%%%%%%%%%%%%%%%%%%%%%%%%%%%%%%%%%%%%%%%%
Since the $\Kbar N$ interaction
has some exoergic channels %reactions
such as $\Kbar N\to \pi Y$,
we need to include additional terms:
\begin{equation}
 \sigtot^{\Kbar N}=\sigbw+\sigel+\sigch+\sigpiY+\sigsS+\sigtS,
\label{eq:xakn}
\end{equation}
where \sigch~ and \sigpiY~ denote $t$-channel charge exchange reaction
  and $t$-channel hyperon production cross sections
  which are also fixed by the requirement that
  the sum of $t$-channel contributions and
  Breit-Wigner contributions reproduce experimental data.
%
Breit-Wigner formula enables us to calculate
  experimentally unmeasured cross sections
  such as $\rho N \to \Lambda K$.
For the calculation of \sigbw,
we include hyperon resonances,
      $R=\Lambda(1405)\sim\Lambda(2110)$
      and 
      $\Sigma(1385)\sim\Sigma(2030)$.
The total and elastic cross sections for $\Kbar N$ interactions
 used in JAM are fit to the date~\cite{PDG96}.


%%%%%%%%%%%%%%%%%%%%%%%%%%%%%%%%%%%%%%%%%%%%%%%%%%%%%
% kbarN -> pi Y and charge exchange cross sections
%\FIGxkny
%%%%%%%%%%%%%%%%%%%%%%%%%%%%%%%%%%%%%%%%%%%%%%%%%%%%%
The symbol $\sigma_{\pi Y}(s)$ in Eq.(\ref{eq:xakn})
 is the sum of $t$-channel pion hyperon production
cross sections $\Kbar N\to \pi Y$, $Y=\Lambda,\Sigma$.
The cross section for the inverse processes such as $\pi Y\to  \Kbar N$ are
calculated using the detailed balance formula.


%%%%%%%%%%%%%%%%%%%%%%%%%%%%%%%%%%%%%%%%%%%%%%%%%%%%%
% KN total and elastic
%\FIGxkaon
%%%%%%%%%%%%%%%%%%%%%%%%%%%%%%%%%%%%%%%%%%%%%%%%%%%%%
$KN$ incoming channel cannot form any $s$-channel resonance
 due to their quark contents. Therefore
the total cross section can be written within our model as follows, 
\begin{equation}
 \sigtot^{KN}=\sigtR+\sigel+\sigch +\sigtS,
\end{equation}
where \sigtR~ is $t$-channel resonance formation cross section.
%
%
In the present version of JAM,
only $K N\to K\Delta$, $K N\to K(892)N$ and $K N \to K(892)\Delta$
are explicitly fitted to experimental data~\cite{CernHera}.

%%%%%%%%%%%%%%%%%%%%%%%%%%%%%%%%%%%%%%%%%%%%%%%%%%%%%
% KN inelastic
%\FIGxkaonD
%%%%%%%%%%%%%%%%%%%%%%%%%%%%%%%%%%%%%%%%%%%%%%%%%%%%%




In meson-meson scattering,
 we also apply the same picture as that in meson-baryon collisions:
\begin{equation}
 \sigtot=\sigbw + \sigtR + \sigel + \sigsS + \sigtS.
\end{equation}
%
%
%
%
The difference between experimental inelastic cross section
and resonance cross sections at energies above resonance region
for the meson-baryon and meson-meson collisions
are attributed to the string formation cross section
where $1/\srt$ energy dependence of \sigsS~ is used~\cite{rqmd2}.

For the cross sections for which no experimental data are available,
we calculate the total and elastic cross sections by using the
additive quark model~\cite{rqmd1,urqmd,goulianos}
\begin{eqnarray}
  \sigma_{tot} &=& \sigma_{NN}\left({n_{1}\over3}\right)\left({n_2\over3}\right)
                 \left(1-0.4{n_{s1}\over n_1}\right)
                 \left(1-0.4{n_{s2}\over n_2}\right) ~, \\
  \sigma_{el} &=& \sigma_{tot}^{2/3}
               (\sigma_{el}^{NN}/ \sigma_{NN}^{2/3}) ~,
\end{eqnarray}
where $\sigma_{NN}$, $\sigma_{el}^{NN}$ express nucleon-nucleon total
and elastic cross sections and $n_i$,$n_{si}$ are the 
number of quarks and $s$-quarks contained in the hadron respectively.
This expression works well above resonance region where the cross section
becomes flat.

For the $t$-channel resonance production cross sections \sigtR,
we do not fit experimental data explicitly in this work
except for $NN$ reaction and one and two pion productions in $KN$ reaction,
because of the vast body of the possibilities for the final states.
 Instead, we simply determine the outgoing resonance types
according to the spins $S_3$, $S_4$ in the final state  and phase space
for the production of resonances $R_3$ and $R_4$
\begin{equation}
% P(R_3,R_4) \sim (2S_3+1)(2S_4+1)p_{34}(s)^2 ~.
 P(R_3,R_4) \propto (2S_3+1)(2S_4+1)p_{34}(s)^2 ~.
\end{equation}
where $p_{34}(s)$ denotes the c.m. momentum in the final state.
If the incoming channel involves resonances, their ground state
particles are also considered in the final state.
Once the outgoing resonance types are determined,
we generate masses according to the Breit-Wigner distribution.

For the angular dependence in the processes of
  $t$-channel resonance production \sigtR,
we use
\begin{equation}
  {d\sigtR \over dt} \sim \exp(bt),
\end{equation}
and the slope parameter $b$ for the energy range of
  $\srt>2.17\GeV$
   is parameterized by
\begin{equation}
   b=2.5 + 0.7\log(s/2),
\end{equation}
with invariant mass squared $s$ given in units of GeV$^2$.
We use the same parameterization presented in Ref.~\cite{niita}
 for the energy below $\srt<2.17\GeV$ for the $t$-channel resonance
productions.
The elastic angular distribution is also taken from Ref.~\cite{niita}
 for $\srt<10\GeV$
and from PYTHIA~\cite{pythia} for $\srt>10\GeV$.









%%%%%%%%%%%%%%%%%%%%%%%%%%%%%%%%%%%%%%%%%%%%%%%%%%%%%%%%%%%%%%%%%
\subsection{Soft interactions modeled by string model} \label{sec:string}
%%%%%%%%%%%%%%%%%%%%%%%%%%%%%%%%%%%%%%%%%%%%%%%%%%%%%%%%%%%%%%%%%

At an energy range above $\srt>4-5$GeV,
the (isolated) resonance picture breaks down because
width of the resonance becomes wider and 
the discrete levels get closer.
The hadronic interactions at
the energy range 4-5$<\srt<$10-100GeV where
it is characterized by the small transverse momentum transfer
is called "soft process", and
string phenomenological models
are known to describe the data for such soft interaction well.
The hadron-hadron collision % of a hadron with another hadron
leads to a string like excitation longitudinally.
In actual description of the soft processes,
we follow the prescription adopted in the HIJING model~\cite{hijing}, 
as described below.

In the center of mass frame of two colliding hadrons,
we introduce light-cone momenta defined by
\begin{equation}
p^+ = E+p_z, \qquad p^- = E-p_z\ .
\end{equation}
Assuming that beam hadron 1 moves in the positive z-direction
and target hadron 2 moves negative z-direction,
the initial momenta of the both hadrons are
\begin{equation}
  p_1 = (p_1^+,p_1^-,0_T), \qquad p_2 = (p_2^+,p_2^-,0_T)\ . 
\end{equation}
After exchanging the momentum $(q^+,q^-,\pp_T)$, the momenta will change to
\begin{equation}
       p'_1 = ((1-x^+)P^+,x^-P^-,\pp_T),
 \qquad p'_2 = (x^+P^+,(1-x^-)P^-,-\pp_T), 
\end{equation}
where $P^+=p_1^++p_2^+=P^-=p_1^-+p_2^-=\srt$ (in c.m. frame).
The string masses will be
\begin{equation}
  M_1^2= x^-(1-x^+)s-p^2_T, \qquad M_2^2=x^+(1-x^-)s-p^2_T,
\end{equation}
respectively.
Minimum momentum fractions are
 $x_{min}^+=p_2^+/P^+$ and $x_{min}^-=p_1^-/P^-$.
For light-cone momentum transfer for the non-diffractive events,
 we use the same distribution as that in
DPM~\cite{dpm} and HIJING~\cite{hijing}:
\begin{equation}
  P(x^{\pm})= {(1.0-x^{\pm})^{1.5}\over (x^{\pm2}+c^2/s)^{1/4}}\ ,
\end{equation}
for baryons and
\begin{equation}
  P(x^{\pm})= {1\over (x^{\pm2}+c^2/s)^{1/4}((1-x^{\pm})^2+c^2/s)^{1/4}}\ ,
\end{equation}
for mesons, where $c=0.1 $GeV is a cutoff.
For single-diffractive events, in order to reproduce
experimentally observed mass distribution $dM^2/M^2$,
we use the distribution
\begin{equation}
   P(x^{\pm})={1 \over (x^{\pm2}+c^2/s)^{1/2}}.
\end{equation}


The same functional form as the HIJING model~\cite{hijing}
  for the soft $\pp_T$ transfer at low 
%$\pp_T<p_0$
$p_T<p_0$
is used
\begin{equation}
   f(\pp_T) = \left\{ (p_T^2+c_1^2)(p_T^2+p_0^2)
                     (1+e^{(p_T-p_0)/c_2})  \right\}^{-1} ~,
\end{equation}
where $c_1=0.1\GeVc$, $p_0=1.4\GeVc$ and $c_2=0.4\GeVc$,
to reproduce the high momentum tail of the particles
at energies $E_{lab}=10\sim20$GeV.




The strings are assumed to hadronize via quark-antiquark 
or diquark-antidiquark % (AO)
creation using Lund fragmentation model PYTHIA6.1\cite{pythia}.
Hadron formation points from a string fragmentation
are assumed to be given by
  the yo-yo formation point~\cite{bialas} which is defined by the 
  first meeting point of created quarks.
Yo-yo formation time is about 1fm/c assuming the string
tension $\kappa=1$ GeV/fm at AGS energies.

%+ADD
In the Lund string model, space-time coordinates and energy-momentum
coordinates for the quarks are directly related
 via the string tension~\cite{lund,lund2}.
Let us consider one-dimensional massless $q\bar q$ string in the c.m..
If $x_i^{\pm}=t_i\pm x_{zi}$ denote the light-cone coordinates of the 
$i$th production point, then the light cone momenta
$p_i^{\pm}=E_n\pm p_{zi}$ of the $i$ th rank hadron
which is produced by the energy-momentum fraction $z_i$
from ($i-1$)th string $  p^+_i = z_i p^+_{i-1}$
 are fixed by
\begin{equation}
p_i^+ = \kappa (x_{i-1}^+ - x_i^+),\quad
p_i^- = \kappa (x_i^- - x_{i-1}^-),
\end{equation}
with initial value
\begin{equation}
  x^+_0 = {W\over \kappa},\qquad x^-_0=0,
\end{equation}
where $W$ corresponds to the string initial invariant mass.
Using the relation $p_i^+p_i^-=m_{i\perp}^2$ with
$m_{i\perp}$ being transverse mass of the $i$th hadron,
we have the recursion formulae~\cite{lund}
\begin{equation}
  x_i^+ = (1-z_i)x^+_{i-1}, \quad
  x_i^- = x^-_{i-1} + \left({m_{i\perp}\over \kappa}\right)^2
                   {1-z_i\over z_i}{1\over x^+_i}.
\end{equation}
Therefore, yo-yo formation points are obtained as
\begin{equation}
    x_i^{yoyo}=(x_{i-1}^+,x_i^-), \label{eq:yoyo}
\end{equation}
and constituent formation points
\begin{equation}
    x_i^{const}=(x_i^+,x_i^-). \label{eq:const}
\end{equation}
In RQMD~\cite{rqmd2}, the formation points of hadrons are calculated
as the average of the two $q\bar q$ production point as
\begin{equation}
    x_i^{rqmd}
  =\left({x_i^+ + x_{i-1}^+\over 2},{x_i^-+x_{i-1}^-\over 2}\right).
  \label{eq:rqmd}
\end{equation}
Clearly, one can see that
\begin{equation}
    x_i^{const} < x_i^{rqmd} < x_i^{yoyo} ~.
\end{equation}
As a default, JAM uses yo-yo formation point.
The different formation points can be used by the switch \ttt{mstc(72)}.





\section{Program description}


\subsection{Random number generator}
Default random number generator in JAM\VERSION~ is
taken from PYTHIA. Another random number generator can be used
when the user rewrite the {\tt function rn}.

\subsection{Main subroutines}

The main program has to be supplied by the user.
These three subroutines are to be called by the main program.
\begin{entry}
\itemt{subroutine jaminit(nev,bmin,bmax,dt,nstep,chfram,chbeam,chtarg,cwin)}
\itemt{subroutine jamevt(iev)}
\itemt{subroutine jamfin}
\end{entry}



\begin{verbatim}
subroutine jaminit(nev,bmin,bmax,dt,nstep,chfram,chbeam,chtarg,cwin)
\end{verbatim}
Purpose: to initialize the program.
This subroutine should be called only once before simulate collision event.
 It reads the parameters
from either the initialization file named {\tt fname(1)} specified by user
or the main program.
\medskip

%\begin{description}
\begin{entry}
\iteme{nev:} integer; specifying the number of total events
  to be generated.
\iteme{bmin:} real*8; Minimum impact parameter(fm).
\iteme{bmax:} real*8; Maximum impact parameter(fm).
If ${\tt bmax}<0$, impact parameter is distributed
      according to the squared $b^2$ distribution
       between {\tt bmin} and {\tt bmax},
     while ${\tt bmax}>0$, impact parameter is uniformly distributed
     between {\tt bmin} and {\tt bmax}.
      If $\ttt{mstc(18)} \mbox{(number of impact parameter bin)} >1$,
     impact parameter is distributed by the bin
      $\delta b=(\ttt{bmax}-\ttt{bmin})/\ttt{mstc(18)}$
      and total number of simulation event becomes
          $\ttt{mstc(18)}\times \ttt{nev}$.

\iteme{dt:} real*8: time step size in fm/c.

\iteme{nstep:} integer: total number of time step.

\iteme{chfram:} character*8 : computational frame. Available options are
  \begin{subentry}
    \iteme{'nn':}        nucleon-nucleon center of mass frame.
    \iteme{'cm':}        Total center of mass frame.
    \iteme{'lab':}       Laboratory frame.
    \iteme{'collider':} Collider experiment.
    \iteme{'user':}   User has to specified particle momemtum and
                     coordinate. i.e.  common block \ttt{jamevnt1},
                      \ttt{jamevnt2} and \ttt{jamjet1}
                     should be filled up by the user as a initial condition
                     of simulation.
    \iteme{'box1.00':} Box calculation to simulate infinite matter,where
                       1.00 corresponds to the density in
                        $\rho_{0}$ (normal nuclear matter) unit.
  \end{subentry}

\iteme{chbeam, chtarg:} character*8 : Projectile and target.
 \begin{subentry}
  \iteme{'p':}     proton.
  \iteme{'pbar':}  antiproton.
  \iteme{'n':}     neutron.
  \iteme{'nbar':}  antineutron.
  \iteme{'pi-':}   negative pion.
  \iteme{'pi0':}   neutral pion.
  \iteme{'pi+':}   positive pion.
  \iteme{'k-':}    negative kaon.
  \iteme{'k+':}    positive kaon.
  \iteme{'197Au'}  Gold nucleus,
         where \ttt{197} indicates mass number of nucleus
         and {\tt Au} specifies proton number.
          'H', 'He', 'Li', 'Be',B','C',....... can be used.
  \itemt{\tt '100:50'} total mass number \ttt{100} and \ttt{50} neutrons.
 \end{subentry}

\iteme{cwin:} character*15 : Incident energy with a unit.
 In the case of \ttt{fram}=\ttt{'user'} or \ttt{'box'}, \ttt{cwin}
is used to estimate the hard cross section above the c.m. energy
of 10GeV. So if the c.m. energy of expected two-body collision
among hadrons are blow 10GeV, \ttt{cwin} does not matter, but
should be set below 10GeV.
    Input examples are the follows:
 \begin{subentry}
  \iteme{'100mev':}  incident energy is 100MeV per nucleon.
  \iteme{'100mevc':} incident momentum is 100MeV/c per nucleon.
  \iteme{'100gev':}  incident energy is 100GeV per nucleon.
  \iteme{'100gevc':} incident momentum is 100GeV/c per nucleon.
  \iteme{'200gev':}  total energy $\sqrt{s}$ is 200GeV per nucleon,
                     in the case of ${\tt chframe}={\tt 'collider'}$.
 \end{subentry}
%\end{description}
\end{entry}


\begin{verbatim}
subroutine jamevt(iev)
\end{verbatim}
{\bf Purpose:} to simulate complete one event.
\begin{entry}
 \iteme{iev:} Current event number (input).
\end{entry}

\begin{verbatim} subroutine jamfin \end{verbatim} {Purpose:} to finish up
simulation.





\subsection{Common blocks for event record}

\begin{verbatim}
      parameter (mxv=30000)
      common /jamevnt1/r(5,mxv),p(5,mxv),v(5,mxv),k(11,mxv)
      common /jamevnt2/nv,nbary,nmeson
\end{verbatim}

\begin{entry}
\itemt{nv:}    Line number which is used during the simulation
               i.e., number of line occupied by the current event.
\itemt{nbary:}  Line number of baryons. Baryons are stored in the
                line \ttt{i=1$\sim$ nbary}.
\itemt{nmeson:}  The particles other than baryons  are stored in the
                line \ttt{i=nbary+1$\sim$ nv}.
\end{entry}

\bigskip

\begin{entry}
\itemt{r(1,i) :} x-position (fm).
\itemt{r(2,i) :} y-position (fm).
\itemt{r(3,i) :} z-position (fm).
\itemt{r(4,i) :} time of particle (fm/c).
\itemt{r(5,i) :} Formation time (fm/c) for the hadron which has
                 constituent quarks.  This has same value as {\tt r(4,i)},
                 if the particle is newly produced particles.
\medskip
\itemt{p(1,i):} x-momentum (GeV/c).
\itemt{p(2,i):} y-momentum (GeV/c).
\itemt{p(3,i):} z-momentum (GeV/c).
\itemt{p(4,i):} kinetic energy $=\sqrt{p_i^2 + m_i^2}$.
\itemt{p(5,i):} mass of particle (GeV/c$^2$).

\medskip
\itemt{v(1,i):} production vertex point, x-position (fm).
\itemt{v(2,i):} production vertex point, y-position (fm).
\itemt{v(3,i):} production vertex point, z-position (fm).
\itemt{v(4,i):} production vertex point, time (fm/c).
\itemt{v(5,i):} life time of the particle in fm/c if it is unstable.

\medskip

\itemt{k(1,i):} status code KS, which gives the current status of
	            the particle stored in the line.
\begin{subentry}
% \itemt{$<0$:}  particle entries with negative status codes are only
%             virtually present and become active only after a certain
%             formation time.
 \itemt{$<0$:}  particle entries which have some constituent quarks.
            Other virtual sea quarks are formed after a certain
            formation time.
 \itemt{$> 11$:} no longer exist. i.e., absorbed particles.
 \itemt{$=1$:} stable particle.
 \itemt{$=2$:} resonance.
 \itemt{$=3$:} string.
 \itemt{$=4$:} parton.
 \itemt{$=5$:} nuclear cluster.
 \itemt{$=-1$:} virtual stable particle (not resonance).
 \itemt{$=-2$:} virtual unstable particle (resonance).
 \itemt{$=-11$:} virtual stable particle but constituent quark can interact.
 \itemt{$=-12$:} virtual resonance particle but constituent quark can interact.
 \itemt{$=-21$:} virtual stable particle but constituent di-quark can interact.
 \itemt{$=-22$:} virtual resonance but constituent di-quark can interact.
 \itemt{$=-31$:} virtual stable particle but constituent 2quarks
                 in a baryon can interact.
 \itemt{$=-32$:} virtual resonance but constituent 2quarks in a baryon
                 can interact.
\end{subentry}

\itemt{k(2,i):} KF particle code i.e. Particle Data Group numbering scheme.
\itemt{k(3,i):} 1000*{\tt mste(1)}+{\tt mste(2)}.
               See section \ref{sec:swi} for the meaning of the
               values \ttt{mste(1)} and \ttt{mste(2)}.
\itemt{k(4,i):} origin of production.
       In the case of collision: 1000*{\tt mste(22)}+{\tt mste(24)},
       while produced from decay: {\tt mste(24)}=KF code of mother particle.
\itemt{k(5,i):} collision counter (internal use to avoid second collision).
\itemt{k(6,i):} multi-step history of collisions.
\itemt{k(7,i):} number of collision suffered so far (initial value is
               set to be  1 for projectile and -1 for target respectively.
\itemt{k(8,i):} identify the same ensemble among test particles.
\itemt{k(9,i):} three times the baryon number.
\itemt{k(10,i):} minimum line number of color flow connection
    in the case of parton. for hadrons this is zero.
\itemt{k(11,i):} maximum line number of color flow connection
    in the case of parton. for hadrons this is zero.
\end{entry}





\subsection{Common blocks for Particle Data}

The following common blocks which contain particle
data are essentially the same as 
 PYTHIA6.1 program~\cite{pythia} with some modifications.

\begin{verbatim}
      common/pydat2/kchg(500,7),pmas(500,4),parf(2000),vckm(4,4)
      common/pydat3/mdcy(500,3),mdme(4000,3),brat(4000),kfdp(4000,5)
\end{verbatim}
{\bf Purpose:} to give particle/parton  data. Particle data is
 stored by compressed code KC that can be calculated by calling function
  {\tt kc=jamcomp(kf)} where {\tt kf} corresponds to the particle flavor code.
I refer users to the original PYTHIA manual~\cite{pythia}
   for detailed description.


\begin{entry}
\item[{\tt kchg(kc,1)}:] three times particle charge.
\item[{\tt kchg(kc,2)}:] color information.
\item[{\tt kchg(kc,3)}:] particle/antiparticle distinction.
\item[{\tt kchg(kc,4)}:] PDG code.
\item[{\tt kchg(kc,5)}:] particle family ID.
\item[{\tt kchg(kc,6)}:] three times baryon number.
\item[{\tt kchg(kc,7)}:] strangeness.
\medskip
\item[{\tt pmas(kc,1)}:] mass in GeV.
\item[{\tt pmas(kc,2)}:] total decay width in GeV.
\item[{\tt pmas(kc,3)}:] deviation of BW.
\item[{\tt pmas(kc,4)}:] the average life time in mm.
\medskip
\item[{\tt mdcy(kc,1)}:] switch to tell whether a particle may be allowed
                         to decay or not.
     \begin{subentry}
        \itemt{$=0$:} the particle is not allowed to decay.
        \itemt{$=1$:} the particle is allowed to decay.
     \end{subentry}
\item[{\tt mdcy(kc,2)}:] entry point into the decay channel table
                         for compressed code kc.
                        It is 0, if no decay channels have been defined.
\item[{\tt mdcy(kc,3)}:] total number of decay channels.
\medskip
\item[{\tt mdme(idc,1)}:] on/off switch for individual decay channel.
     \begin{subentry}
       \itemt{$=0$:} channel is switched off.
       \itemt{$=1$:} channel is switched on.
     \end{subentry}

\item[{\tt mdme(idc,2)}:] matrix element in decay.
     \begin{subentry}
       \itemt{$=0$:} momentum of particles are determined from
           phase space.
       \itemt{$=1$:} $\omega$ and $\phi$ decays into three pions.
     \end{subentry}
\item[{\tt mdme(idc,3)}:]  relative angular momentum for each decay channel
                           in two-body decay.
\end{entry}
For the detailed meaning of \ttt{mdme(idc,1),mdme(idc,2),mdme(idc,3)},
  See the original manual Ref.~\cite{pythia}.


\subsection{Further common blocks}

For completeness, we list here the common blocks which contain
 include file {\tt jam1.inc}.

\begin{verbatim}
      common/jamjet1/vq(10,mxv),kq(2,mxv)
\end{verbatim}
{\bf Purpose:} to store parton properties or decay orientation.

 For the strings,
\begin{entry}
\itemt{kq(1,i):} flavor code of    quark.
\itemt{kq(2,i):} flavor code of    diquark (anti-quark).
\itemt{vq(1,i):} $p_x$ (GeV/c) of parton     (quark).
\itemt{vq(2,i):} $p_y$ (GeV/c) of parton     (quark).
\itemt{vq(3,i):} $p_z$ (GeV/c) of parton     (quark).
\itemt{vq(4,i):} energy(GeV)  of parton  (quark).
\itemt{vq(5,i):}  mass (GeV/c$^2$) of parton    (quark).
\itemt{vq(6,i):} $p_x$ (GeV/c) of parton     (diquark/anti-quark).
\itemt{vq(7,i):} $p_y$ (GeV/c) of parton     (diquark/anti-quark).
\itemt{vq(8,i):} $p_z$ (GeV/c) of parton     (diquark/anti-quark).
\itemt{vq(9,i):}  energy (GeV)of parton  (diquark/anti-quark).
\itemt{vq(10,i):} mass of(GeV/c$^2$) parton    (diquark/anti-quark).
\end{entry}

\medskip
For the resonances, \ttt{kq(j,i)} and \ttt{vq(j,i)} contain the following
information:
\begin{entry}
\itemt{kq(1,i):} $=999999$.
\itemt{kq(2,i):} $=0$.
\itemt{vq(1,i):} $\beta_x$ (GeV/c) in two-body c.m..
\itemt{vq(2,i):} $\beta_y$ (GeV/c)  in two-body c.m..
\itemt{vq(3,i):} $\beta_z$ (GeV/c)  in two-body c.m..
\itemt{vq(4,i):} $\gamma$ in two-body c.m..
\itemt{vq(5,i):} $\phi$  in two-body c.m.
\itemt{vq(6,i):} $\theta$ in two-body c.m.
\itemt{vq(7,i):} $=0$.
\itemt{vq(8,i):} $=0$.
\itemt{vq(9,i):} $=0$.
\itemt{vq(10,i):} $=0$.
\end{entry}

\begin{verbatim}
      common/jamsave1/rcp(5,6),pcp(5,6),vcp(5,6),kcp(11,6)
      common/jamsave2/vqcp(10,6),kqcp(2,6)
\end{verbatim}
{\bf Purpose:} to save two-body collision information.
Before performing two-body collision or decay,
 particle information is stored in \ttt{common jamsave1, jamsave2}.
Meaning is the same as corresponding vectors.


\begin{verbatim}
      parameter (mxcoll=50000)
      common/jamcol/coll(6,mxcoll),icoll(3,mxcoll),mentry
\end{verbatim}
{\bf Purpose:} to store collision matrix.
\begin{entry}
\itemt{mxcoll:} Maximum collision number in each time step.
\itemt{mentry:} number of entry of collision pair which will collide.
\itemt{icoll(1,i):} line number of particle 1 specified by collision entry i.
\itemt{icoll(2,i):} line number of particle 2 specified by collision entry i.
\itemt{icoll(3,i):} cell number specified by collision entry i
                  used in box calculation.
\itemt{coll(1,i):} contain time of collision in computational frame (fm/c).
\itemt{coll(2,i):} contain total cross section (mb).
\itemt{coll(3,i):} contain elastic cross section (mb).
\itemt{coll(4,i):} contain impact parameter squared (fm$^2$).
\end{entry}

\begin{verbatim}
      common/jamcell/rcell(3,27)
\end{verbatim}
{\bf Purpose:} to store cell for box calculation.
\begin{entry}
\itemt{rcell(i,j):} cell in the case of BOX calculation.
\end{entry}






\subsection{The General Switches, Options and Parameters}\label{sec:swi}

\begin{verbatim}
      common/jamdat1/mstc(200),parc(200),mstd(200),pard(200)
\end{verbatim}
{\bf Purpose:} to contain input parameters, default values and options
 in \ttt{mstc} and \ttt{parc}. The  \ttt{mstd} and \ttt{pard}
 give event information. Default values are indicated by (D=),
while variables that used internally are denoted by (I).

\begin{entry}
\itemt{mstc(1) :} (D=19780503) random seed.
\itemt{mstc(2) :} (D=1) total number of event to be generated.
\itemt{mstc(3) :} (D=1) total number of time step.
           \ttt{mstc(3)$\times$parc(2)} is a time (fm/c) in the case of
          \ttt{mstc(44)=1}.
\itemt{mstc(4) :} (D=2) computational frame.
 \begin{subentry}
        \itemt{$=0$:} laboratory system.
        \itemt{$=1$:} overall c.m. system.
        \itemt{$=2$:} nucleon-nucleon c.m. system.
        \itemt{$=3$:} collider.
       \itemt{$=10$:} box.
       \itemt{$=100$:} user defined initial condition.
 \end{subentry}
\itemt{mstc(5) :} number of test particles.
\itemt{mstc(6) :} (D=0) treatment of AA collisions.
   \begin{subentry}
      \itemt{$<-101$ :} Collisions are generated according to the 
                     mean free path argument. This option can only
                     be used with baryon-nucleus collisions.
      \itemt{$=-111$ :} mean free path method 1 with uniform matter density.
      \itemt{$=-112$ :} mean free path method 1 with Fermi density.
      \itemt{$=-121$ :} mean free path method 2 with uniform matter density.
      \itemt{$=-122$ :} mean free path method 2 with Fermi density.
      \itemt{$=-131$ :} mean free path method 3 with uniform matter density.
      \itemt{$=-132$ :} mean free path method 3 with Fermi density.

       where, mean free path method 1 is defined as following~\cite{KK}:\\
       We determine the collision point according to the probability
       distribution $dP=\exp(-dr/\lambda)/\lambda dr$,
       with $\lambda=1/(P_p+P_n+P_d)$,
       $P_p=\sigma_p\rho_p$, $P_n=\sigma_n\rho_n$
       $P_d=\Gamma/v\gamma$.

       In method 2 and 3, collision probabilities are
       calculated from 
       $dP=\sigma \rho(\rr)dz$ and $dP=\sigma \rho(\rr)v dt$ respectively
       in order to account for the defuseness of the nuclear density.
       Note that \ttt{mstc(6)=-112} would not give correct collision
       probability.


      \itemt{$=-3$ :} Glauber type multiple collision for AA collision,
                      sequential collisions are generated according to
                      the additive quark cross section.
      \itemt{$=-2$ :} Glauber type multiple collision for AA collision,
                      but if the scattering of baryons
                      which have only one-constituent quark are not 
                      considered.
      \itemt{$=-1$ :} Glauber type multiple collision for AA collision,
                      Fritiof type calculation.
      \itemt{$=-21,-22,-23$ :} same as \ttt{mstc(6)=-1,-2,-3},
      but after performing
       all predicted NN collisions and fragment strings,
       final state hadron interaction are simulated (not work now).

      \itemt{$=0$ :} Cascade mode.
      \itemt{$=1$ :} no collision and no mean field.
      \itemt{$=2$ :} BUU mode: Soft Skyrm force and no collision.
      \itemt{$=4$ :} BUU mode: Hard Skyrm force and no collision.
      \itemt{$=12$ :} BUU mode: Soft Skyrm force.
      \itemt{$=14$ :} BUU mode: Hard Skyrm force.
   \end{subentry}

\itemt{mstc(7) :} (D=0) selection of a reaction type. This version only
    allows $=0$.
\comment{
      = 0 : light and heavy-ion collisions as well as proton, anti-proton
           ,kaon, pion .... induced.
      = 41: Stopped K- reaction  (Not implemented now).
      = 42: Stopped Xi- reaction  (Not implemented now).
      = 42: Stopped Omega- reaction (Not implemented now).
      = 43: (K-,K+)                  (Not implemented now).
      = 44: Charm nucleus production (Not implemented now).
}

\itemt{mstc(8) :} (D=0) Job mode (debug mode).
\begin{subentry}
\itemt{$=0$:} almost no information are printed. This is useful for large JOB.
\itemt{$=1$:} simple information are printed.
\itemt{$=2$:} detailed information are printed. After every collision and decay
         various errors are checked (momentum conservation ....).
\itemt{$=3$:} more detailed information are printed.
\itemt{$=4$:} more and more detailed information are printed.
\end{subentry}

\itemt{mstc(9)  :}(D=27) contains number of cells for box simulation.
\itemt{mstc(10) :} (D=1) number of impact parameter bin.

\itemt{mstc(11) :} (D=1) check on possible errors.
  \begin{subentry}
        \itemt{$=1$:} No.
        \itemt{$=2$:} stop program after writing out error message.
  \end{subentry}

\itemt{mstc(12)  :} (D=30) number of errors that are printed.

\itemt{mstc(13)  :} (D=1) printing of warning messages.
  \begin{subentry}
        \itemt{$=0$:} no warning is written.
        \itemt{$=1$:} \ttt{mstc(12)} warnings are printed.
        \itemt{$=2$:} all warnings are printed.
  \end{subentry}

\itemt{mstc(14)  :} (D=30) number of warnings that are printed.

\itemt{mstc(15)  :} not used.

\itemt{mstc(16)  :}(D=0) display particles on any terminal during the 
                   simulation.  Time step for display can be controlled
                   by \ttt{parc(7)} and scale of display for \ttt{parc(6)}.
\begin{subentry}
       \itemt{$=0$ :} no display.
       \itemt{$=1$ :} configuration space display (all hadrons).
       \itemt{$=2$ :} configuration space display (baryons only).
       \itemt{$=3$ :} configuration space display (mesons only).
       \itemt{$=11$:} momentum space display (all hadrons).
       \itemt{$=12$:} momentum space display (baryons only).
       \itemt{$=13$:} momentum space display (mesons only).
\end{subentry}

\itemt{mstc(17):} (D=0) switch in the case of elementary
           ($pp$,$\pi p$...)collisions.
       \itemt{$=0$ :} Inelastic as well as  elastic collisions are
                      generated.
       \itemt{$=1$ :} only inelastic collisions are generated.

\itemt{mstc(18)-mstc(19)  :} not used.
\itemt{mstc(20) :}(D=0) flag for call of \ttt{subroutine jamsetp}.
   if \ttt{jamsetp} was called \ttt{mstc(20)$=1$}.

\itemt{mstc(21) :}(D=0)(I) number of call of \ttt{subroutine jaminit}.
                   This is mainly used for the initialization
                   procedures which is necessary only once.

%---------------------------------------------------

\itemt{mstc(22) :} (I) starting point of KC code for $N^*$ resonance.
\itemt{mstc(23) :} (I) end point of KC code for $N^*$ resonance.
\itemt{mstc(24) :} (I) starting point of KC code for $\Delta^*$.
\itemt{mstc(25) :} (I) end point of KC code for $\Delta^*$.
\itemt{mstc(26) :} (I) starting point of KC code for $\Lambda^*$.
\itemt{mstc(27) :} (I) end point of KC code for $\Lambda^*$.
\itemt{mstc(28) :} (I) starting point of KC code for $\Sigma^*$.
\itemt{mstc(29) :} (I) end point of KC code for $\Sigma^*$.
\itemt{mstc(30) :} (I) starting point of KC code for $\Xi^*$.
\itemt{mstc(31) :} (I) end point of KC code for $\Xi^*$.
\itemt{mstc(32) :} (I) starting point of KC code for mesons.
\itemt{mstc(33) :} (I) end point of KC code for mesons.

%-----------------------------
% Input and output file names
%-----------------------------
\itemt{mstc(34)-mstc(36) :} not used.

\itemt{mstc(36) :}(D=1) I/O number for input configuration file:
                   \ttt{fname(1)}.
\itemt{mstc(37) :}(D=2) I/O number for output file : \ttt{fname(2)} that
              print out some information on simulation conditions.
\itemt{mstc(38) :}(D=3) I/O number for output of error message. \ttt{fname(3)}.
\itemt{mstc(39) :}(D=4) I/O number for multi run control file. \ttt{fname(4)}.
\itemt{mstc(40) :}(D=8) I/O number for \ttt{fname(5)}.

\medskip

%----------------------------------------------------------------------------
\itemt{mstc(41)  :} (D=1) switch for resonance decay after simulation.
  \begin{subentry}
    \itemt{$=0$} all resonances ($\Delta$,$\rho$....) will remain
                 after the end of simulation,
                 if their decay time is longer than simulation time.
    \itemt{$=1$} all resonances ($\Delta$,$\rho$....).
                 are forced to decay at the end of simulation,
                 if they remain undecayed.
    \end{subentry}

\itemt{mstc(42)  :} (D=1): option to turn off the weak decay of $\pi^0$,
 $\Lambda$.....
    \begin{subentry}
        \itemt{$=0$:} allow weak decay after simulation.
        \itemt{$=1$:} weak decay is turned off.
    \end{subentry}
\itemt{mstc(43)  :} (D=1) switch for inclusion of Fermi motion of nucleons
                    in nucleus.
    \begin{subentry}
       \itemt{$=0$ :} without Fermi motion of nucleons.
       \itemt{$=1$ :} Fermi motion is included.
    \end{subentry}

\itemt{mstc(44)  :} (D=1) switch for the end of calculation.
    \begin{subentry}
       \itemt{$=1$ :} simulation is stopped at the time=
         \ttt{mstc(3)$\times$parc(2)} which is
         specified by the user.
       \itemt{$=2$ :}  simulation is continued until all collisions
        and decays are not expected.
    \end{subentry}

\itemt{mstc(45)  :} (D=0) switch for the deuteron coalescence.
    \begin{subentry}
       \itemt{$=0$ :} no.
       \itemt{$=1$ :} deuterons are formed at the end of simulation.
    \end{subentry}

\itemt{mstc(46)  :} (D=1) option for the treatment of cascading 
   in the case of more than one parallel run.
    \begin{subentry}
       \itemt{$=1$ :} parallel ensemble method.
       \itemt{$=2$ :} full ensemble method in which all particles
                      among test particles can collide
                      with reduced cross section
                      $\sigma_{tot}/\tt{mstc(5)}$.
    \end{subentry}


%------------------------
% Collisions 51-99
%------------------------

\itemt{mstc(51)(D=3)  :} switch for two body  collisions.
    \begin{subentry}
           \itemt{$=0$ :} only baryon-baryon collisions.
           \itemt{$=1$ :} only baryon-baryon  and  meson-baryon collisions.
           \itemt{$=2$ :} hadron-hadron collisions (no parton collisions).
           \itemt{$=3$ :} hadron-parton and  hadron-hadron collisions.
           \itemt{$=4$ :} all (not implemented in this version).
    \end{subentry}

\itemt{mstc(52) :} (D=5) switch for cascading method.
    Collision frame is chosen to be the two particle center of mass frame,
    but there are some option for the selection of ordering
    of collision time and point~\cite{zpc,bin1}.
 \begin{subentry}
    \itemt{$=2$:} collisions and decays are ordered according to the
                  average of the two collision times.
    \itemt{$=3$:} the earlier collision time of the two collision times
                  in the computational frame are taken to be the ordering time.
    \itemt{$=4$:} the later collision time of the two collision times
                  in the computational frame are taken to be the ordering time.
    \itemt{$=5$:} the collision space point is the midpoint of
                  the two particles in two-body c.m.frame.
    \itemt{$=11$:} all particles have always same time whenever collision
                   or decay happened; all particles are propagated up
                   to this time.
  \end{subentry}

\itemt{mstc(53)(D=1) :} switch for treatment of particles which are within a
                         formation time.
 \begin{subentry}
   \itemt{$=1$:} collision can happen with $\sigma_{hh}=$ additive quark
                 cross section within a formation time.
    \itemt{$=2$:}any collision can not happen within a formation time.
    \itemt{$=3$:}no formation time for hadrons which has constituent quarks.
 \end{subentry}

\itemt{mstc(54) :} (D=1)
  \begin{subentry}
    \itemt{$=1$ :} avoid first collisions between nucleons among same nucleus
                   that has not yet collided.
   \itemt{$=0$ :}  allow all collisions.
  \end{subentry}

\itemt{mstc(55) :} (D=0) frozen resonance option.
  \begin{subentry}
   \itemt{$=0$ :} off.
    \itemt{$=1$ :} this keeps unstable particles stable
                   during the simulation and only decay
                   after the end of simulation.
  \end{subentry}

\itemt{mstc(56) :} (D=1) switch for Pauli-blocking in collisions and decays.
  \begin{subentry}
     \itemt{$0$ :} no Pauli-blocking.
     \itemt{$1$ :} blocking factor is calculated by the one-body phase space
                   distribution function using Gaussian wave-packet as
                   $f_{i}=\sum_{i \neq j}
    =8\exp[-{(\rr_i-\rr_j)^2\over 2L} - {2L(\pp_i-\pp_j)^2\over \hbar^2} ]$,
                  where $L$ is defined by the parameter \ttt{parc(15)}.
   \itemt{$2$ :} blocking factor is calculated by Fusimi function as
                 $f_i=\sum_{i \neq j}|<\phi_i|\phi_j>|
       =\sum_{i \neq j}\exp[
          -{(\rr_{i}-\rr_{j})^2\over 4L}-{(\pp_i-\pp_j)^2L\over \hbar^2}]$,
                where $L$ is defined by the parameter \ttt{parc(15)}.
  \end{subentry}

\itemt{mstc(57) :} (D=0) not used.
\comment{
         =0: No
         =1: Recalculate momenta of the colliding particles in order to
             recover total energy conservation in case of potential forces act.
}
%------------------------
% resonance
%------------------------
\medskip

\itemt{mstc(61)(D=3) :} switch of baryon resonance decay angular distribution.
\begin{subentry}
  \itemt{$=0$ :} isotropic decay.
  \itemt{$=1$ :} anisotropic with Gaussian $p_t$, width of Gaussian is
                 \ttt{parc(43)}.
  \itemt{$=2$ :} anisotropic + isotropic with Gaussian $p_t$, 
                 width of Gaussian is \ttt{parc(43)}.
  \itemt{$=3$ :} anisotropic with Gaussian $p_t$ distribution,
                 but  ${d\sigma \over dp_t}$ is reinterpreted as 
                ${d\sigma \over p_rd\theta}$,
                where $p_r$ is relative momentum
                in rest frame of decaying particle.
                At high energies and forward angles, these become the same.
\end{subentry}

\itemt{mstc(62)(D=22) :} option for detailed balance formula.
  \ttt{isw1=mod(mstc(62)/10,10)} is used for the choice of
   Breit-Wigner function, while \ttt{isw2=mod(mstc(62),10)}
   is used to specify different formula.
\begin{subentry}
   \itemt{isw1$=1$:} non-relativistic Breit-Wigner function
                     $F(m)dm={\Gamma/2 \over (m-m_0)^2 + \Gamma^2/4}dm$
                     is used.
   \itemt{isw1$=2$:} relativistic Breit-Wigner function
                   $F(m)dm={2mm_0dm\Gamma \over (m^2-m_0^2)^2 + (m_0\Gamma)^2}$
                     is used.
   \itemt{isw1$=3$:} relativistic Breit-Wigner function
                    $F(m)dm={2m^2dm\Gamma \over (m^2-m_0^2)^2 + (m\Gamma)^2}$
                     is used.
   \itemt{isw2$=0$:} constant decay width is used.
   \itemt{isw2$=1$:} no-momentum dependence in integration~\cite{detbal1}.
   \itemt{isw2$=2$:} momentum dependence in integration.
   \itemt{isw2$=3$:} P.Danielevicz and G.F.Bertch formula
                         ~\cite{detbal1,detbal2}.
\end{subentry}

\itemt{mstc(63) :}(D=3) option for the formula of $\Delta(1232)$ decay width.
 \begin{subentry}
  \itemt{$=0$:} no decay(frozen delta).
  \itemt{$=1$:} formula by Frankfurt group.
  \itemt{$=2$:} formula by Giessen group.
  \itemt{$=3$:} formula in Ref.~\cite{dwid1}.
  \itemt{$=4$:} formula by Kitazoe \etal Ref~\cite{Kitazoe}.
 \end{subentry}

\itemt{mstc(64) :}(D=3) option for the method of sampling resonance mass
                        distribution in non-strange baryon-baryon collisions.
 \begin{subentry}
  \itemt{$=0$:} mass is sampled according to the probability $p(m)dm= dm/m$.
  \itemt{$=1$:} mass is selected by Breit-Wigner,
                but individual resonance states are selected from only spin.
  \itemt{$=2$:} mass and individual resonance states are selected by
                the probability obtained from integrated Breit-Wigner.
  \itemt{$=3$:} same as 2 but width in Breit-Wigner function
                is momentum dependent.
 \end{subentry}

\itemt{mstc(65) :} (D=1) option for the resonance width for decay.
 \begin{subentry}
         \itemt{$=1$ :} momentum dependent width.
         \itemt{$=0$ :} constant width.
 \end{subentry}

\itemt{mstc(66)(D=0) :} option for the treatment of s-wave pion production
   in nucleon-nucleon collisions. 
 \begin{subentry}
    \itemt{$=0$ :} s-wave one-pion production is effectively simulated by
        $NN \to NN(1440)^*$.
    \itemt{$=1$ :} include s-wave pion productions explicitly,
                   as $NN\to NN\pi$.
 \end{subentry}

\itemt{mstc(67),mstc(68),mstc(69),mstc(70) :} (D=2,2,2,2)
     selection of the angular distribution for
     resonance production scattering in $BB$(\ttt{mstc(67)}),
     $MB$(\ttt{mstc(68)}), $MM$(\ttt{mstc(69)}) and $\Bbar B$(\ttt{mstc(70)})
     collisions at energy above $\srt=2.17$GeV.
 \begin{subentry}
    \itemt{$=2$ :} angular dependence is sampled by $\exp(bt)$ distribution.
                   See \ttt{parc(44)-parc(46)}.
    \itemt{$=4$ :} angular dependence is sampled by Gaussian distribution:
                   $\exp[-(p_T/\ttt{parc(47)})^2]$.
    \itemt{$=5$ :} angular dependence is sampled by Gaussian plus
                    exponential distribution:
        $\exp[-p_T/\ttt{parc(48)})]+\ttt{parc(49)}\exp[-p_T/\ttt{parc(47)})]$.
 \end{subentry}
%----------------------
% soft interactions
%----------------------
\medskip

\itemt{mstc(71)(D=1) :} switch for diffractive reaction in hh collisions.
 \begin{subentry}
    \itemt{$=0$ :} diffractive scattering off.
    \itemt{$=1$ :} on.
 \end{subentry}

\itemt{mstc(72)(D=2) :} switch for hadron formation point from string decay.
 \begin{subentry}
    \itemt{$=1$:} constituent production formation point.
    \itemt{$=2$:} yo-yo formation point.
    \itemt{$=3$:} mean value of yo-yo point and constituent production point.
    \itemt{$=4$:} constant formation time specified by \ttt{parc(55)}.
 \end{subentry}

\itemt{mstc(73):}(D=2) choice of baryon production model in string decay
    (same as \ttt{mstj(12)} in PYTHIA).
 \begin{subentry}
   \itemt{$=0$ :} no baryon-antibaryon pair production at all.
   \itemt{$=1$ :} diquark-antidiquark pair production allowed; diquark
                  treated as a unit.
   \itemt{$=2$ :} diquark-antidiquark pair production allowed;
           with possibility for diquark breaking according to 'popcorn' scheme.
   \itemt{$=3$ :} as =2, but with improved SU(6) treatment.
   \itemt{$=4$ :} as =3, but also suppressing diquark vertices with low Gamma
                  values.
   \itemt{$=5$ :} revised popcorn model. Independent of \ttt{parj(3-7)}.
                  Depending on \ttt{parj(8-10)}. Including the same kind of
                 suppression as =4.
 \end{subentry}

\itemt{mstc(74)(D=1) :} switch for dipole-approximated QCD radiation
                    of the string system in soft interactions.

 \begin{subentry}
   \itemt{$=0$ :} off.
   \itemt{$=1$ :} on.
 \end{subentry}

\itemt{mstc(75)(D=1) :} switch for the $p_t$ kick due to soft interactions
                 (\ttt{ihpr2(5)} in HIJING).
 \begin{subentry}
   \itemt{$=0$ :} $p_t$ kick is sampled by Gaussian.
   \itemt{$=1$ :} include extra low $p_t$ transfer to the valence quarks
       with parameter \ttt{parc(67-68)}.
 \end{subentry}

\itemt{mstc(76) (D=0) :} treatment string in space-time.
    =0: all strings decay into hadrons immediately taking
       the formation time into account. 
    =1: strings from hard scattering
       are explicitly propagated.
    =2: strings from soft excitation as well as hard scattering
       are explicitly propagated (see parc(72)) (not work now).


%------------------
%  hard scattering
%------------------
\medskip

\itemt{mstc(81) :} (D=1)  switch for hard scattering.
 \begin{subentry}
    \itemt{$=0$ :} off.
    \itemt{$=1$ :} hard scattering on.
 \end{subentry}

\itemt{mstc(82) :}(D=3) choice of proton structure functions.
 \begin{subentry}
    \itemt{$=1$ :} EHLQ set 1 (1986 updated version).
    \itemt{$=2$ :} EHLQ set 2 (1986 updated version).
    \itemt{$=3$ :} Duke-Owens set 1.
    \itemt{$=4$ :} Duke-Owens set 2.
    \itemt{$=5$ :} CTEQ2M (best MSbar fit)
    \itemt{$=6$ :} CTEQ2MS (singuar at small x)
    \itemt{$=7$ :} CTEQ2MF (flat at small x)
    \itemt{$=8$ :} CTEQ2ML (large $\Lambda$)
    \itemt{$=9$ :} CTEQ2L (best leading order fit)
    \itemt{$=10$ :} CTEQ2D (best DIS fit)
    \itemt{$=11$ :} CRV LO (1992 updated version)
    \itemt{$=12$ :} CTEQ 3L (leading order). 
    \itemt{$=13$ :} CTEQ 3M (MSbar).
    \itemt{$=14$ :} CTEQ 3D (DIS).
    \itemt{$=15$ :} GRV 94L (leading order).  % *
    \itemt{$=16$ :} GRV 94M (MSbar).
    \itemt{$=17$ :} GRV 94D (DIS).
 \end{subentry}

\itemt{mstc(83) :}(D=1) 
 \begin{subentry}
     \itemt{$=1$ :} probability of number of jet from HIJING formalism.
     \itemt{$=2$ :} probability of number of jet from PYTHIA formalism.
 \end{subentry}

\itemt{mstc(84) :}(D=0) Option for nuclear shadowing effect.
 \begin{subentry}
            \itemt{$=0$ :} off.
            \itemt{$=1$ :} on.
 \end{subentry}

\itemt{mstc(85) :}(D=2) choice of scheme for parton collision-time estimate.
 \begin{subentry}
            \itemt{$=1$ :} exponential distribution.
            \itemt{$=2$ :} Gyulassy-Wang distribution.
 \end{subentry}


%
%  BUU part
%
\medskip

\itemt{mstc(101) :}(D=0) option for Coulomb potential.
 \begin{subentry}
            \itemt{$=0$ :} off.
            \itemt{$=1$ :} on.
 \end{subentry}
\itemt{mstc(102) :}(D=1) time step to calculate the boundary of Coulomb.


\medskip
%---------------------------
\subsubsection*{ Event study and analysis}
The following switches are provided for the study of
some observable and time evolution of particle
in the case of \ttt{mstc()}=1.
As a default (D=0), nothing is done.
%---------------------------
%General observable:
% if =2, output every run.
\comment{mstc(151): (D=0) not used (Nuclear cluster analysis)}

\itemt{mstc(152):} (D=0) rapidity spectra $dN/dy$.
\itemt{mstc(153):} (D=0) transverse momentum spectra $1/p_{\perp}dN/dp$.
\itemt{mstc(154):} (D=0) particle multiplicity.
\itemt{mstc(155):} (D=0) flow $d<Px>/dy$, $d<P_t>/dy$.
\itemt{mstc(156):} (D=0) invariant energy distribution of two-body collisions.
\itemt{mstc(157):} (D=0) output sampled ground state.

\medskip
Time dependent studies:
\itemt{mstc(161):} (D=0) all analyses will not be done.
\itemt{mstc(162):} (D=0) count two-body collisions and decays in time.
\itemt{mstc(163):} (D=0) time evolution of directed transverse flow.
\itemt{mstc(164):} (D=0) output time evolution of phase space data.
 \begin{subentry}
   \itemt{$=1$ :} MDRAW~\cite{mdraw} format.
                  Before using this, directory 'DISP' is required
                  in current directory.
   \itemt{$=2$ :} ANGEL~\cite{angel} format.
 \end{subentry}
\itemt{mstc(165):} (D=0) time evolution of particle.
\itemt{mstc(166):} (D=0) time evolution of particle density using small shell.
\itemt{mstc(167):} (D=0) time evolution of particle density
                         using Gaussian smearing.
\itemt{mstc(168):} (D=0) time evolution of temperature written by Toshiki
   \begin{subentry}
    \itemt{$=1$ :} normal temperature fit.
    \itemt{$=2$ :} variable bin (under construction).
   \end{subentry}
\itemt{mstc(169):} (D=0) =1: show how the temperature was fitted,
                    if \ttt{mstc(168)=1}.

\end{entry}

\bigskip
\bigskip

% PARC()
\begin{entry}
\itemt{parc(1)  :} not used.
\comment{
\itemt{parc(1)  :} (D=-1.) max. CPU time to simulate.
            if parc(1)<0, there is no limit.
}
\itemt{parc(2)  :} (D=100.0) time step size for evolution in fm/c.

\itemt{parc(3)  :} (D=0.0)  minimum impact parameter in "fm" \ttt{bmin}.

\itemt{parc(4)  :} (D=0.0)  maximum impact parameter in "fm" \ttt{bmax}.
       If $\ttt{bmax}<0$, impact parameter is distributed
        according to the squared $b^2$
       distribution between \ttt{bmin} and \ttt{bmax},
       while $\ttt{bmax}>0$ , impact parameter is uniformly distributed
       between bmin and bmax.

\itemt{parc(5) :} (D$=-1.0$fm) initial distance $r_z$ between beam and target.
 \begin{subentry}
  \itemt{$< 0.0$:}$r_z = r_1+r_2+6.0$ for $E_{lab}<$ 200MeV,\\
                  $r_z = r_1+r_2+4.0$ for 200MeV$<E_{lab}<$1GeV,\\
                  $r_z = r_1/\gamma_p + r_2/\gamma_t + 2.0$ for $E_{lab}>$1GeV.
  \itemt{$>=0.0$:}$r_z = r_1/\gamma_p + r_2/\gamma_t + \ttt{parc(5)}$.
 where $r_1$ and $r_2$ represent projectile and target radius respectively,
 and $\gamma_p$ and $\gamma_t$ correspond to $\gamma$ factor for
 the projectile and target.
\end{subentry}

\itemt{parc(6):} (D=2.0) scale of display for switch \ttt{mstc(16)}.
\itemt{parc(7):} (D=1.0fm/c) time slice for time dependent analysis.
\itemt{parc(7):} (D=1.0fm/c) time slice for display.

\itemt{parc(8-10):} not used.
%-----------------
% Ground state 20-29
%-----------------

\itemt{parc(11) :} (D=0.54fm) diffuseness parameter in Woods-Saxon distribution
                              for the density of nucleus.
\itemt{parc(12) :} (D=1.124fm) parameter of Woods-Saxon tail.
\itemt{parc(13) :} (D=0.8fm)   minimum distance between identical nucleons
                               for sampling of nucleus.
\itemt{parc(14) :} (D=0.8fm) minimum distance  between $p$ and $n$
                             for sampling of position.
\itemt{parc(15) :} (D=2.0 fm$^2$) width of Gaussian wave packet,
                  here definition of $L$ is
               $ \phi_i(r)={1\over (2\pi L)^{3/4}}
                 \exp[-{(r-R_i)^2 \over 4L} + {i \over \hbar}r\cdot P_i] $.

\itemt{parc(16)-parc(20) :} not used.
%----------------------
% Constant parameters
%----------------------
\itemt{parc(21) :} (D=0.168$1/\fm^3$) normal nuclear matter density $\rho_0$.
\itemt{parc(22) :} (D=$-16.0$MeV)  binding energy of nuclear matter.
\itemt{parc(23) :} (D=0.001439767) Coulomb constant.
\itemt{parc(24) :} (D=0.93960GeV$^2$) neutron masses.
\itemt{parc(25) :} (D=0.93830GeV$^2$) proton mass.
\itemt{parc(26) :} (D=0.13500GeV$^2$) $\pi^0$ mass.
\itemt{parc(27) :} (D=0 .13960GeV$^2$) $\pi^+$ mass.
\itemt{parc(28) :} (D=0.938950002GeV$^2$) nucleon mass.
\itemt{parc(29) :} (D=0.1373GeV$^2$) average pion mass.
\itemt{parc(30) :} not used.

%--------------
%  Collisions 31-
%--------------
\medskip

\itemt{parc(31) :} (D=5.0fm) maximum spatial distance for which
            a two-body collision still can occur.

\itemt{parc(32)-parc(38) :} low energy cut off for $hh$ collisions.
\itemt{parc(32) :} (D=55mb) maximum cross section
                            for$ pn$ collisions at low energy.
\itemt{parc(33) :} (D=55mb) maximum cross section
                            for $pp$ collisions at low energy.
\itemt{parc(34) :} (D=200mb) maximum cross section
                             for $BB$ collision at low energy
                            (involving resonance).
\itemt{parc(35) :} (D=200mb) maximum cross section
                             for $MB$ collisions at low energy.
\itemt{parc(36) :} (D=150mb) maximum cross section
                             for $MM$ collisions at low energy.
\itemt{parc(37) :} (D=350mb) maximum cross section
                             for $\Bbar B$ collision at low energy.
\itemt{parc(38) :} (D=0.05GeV) kinetic energy cutoff
                               for two-body collisions at low energy.

%------------
% resonance
%------------
\itemt{parc(41) :} (D=0.001GeV) minimum kinetic energy in decays.
\itemt{parc(42) :} (D=0.5fm) production point of daughter from mother in decay.
\itemt{parc(43) :} (D=0.4GeV/c) width of Gaussian $p_t$ distribution
                    for the decay of resonance, see \ttt{mstc(61)}.

\itemt{parc(44),parc(45),parc(46) :}(D=2.5(GeV/c$^{-2}$,0.7,2.0)
                   slope parameter for angular distribution
                   in inelastic $hh$ collisions: $d\sigma/dt=e^{at}$
                   ~\cite{ref:slope} for resonance production
                   in the case of \ttt{mstc(67-70)=2},
                   where slope parameter is parametrized by
        $a=\ttt{parc(44)}+\ttt{parc(45)}*(\log(s)-\log(\ttt{parc(46)}))$.

\itemt{parc(47) :} (D=0.36(GeV/c)$^{-2}$) Gaussian width for the option of
                    \ttt{mstc(67)=4,5}.
\itemt{parc(48) :} (D=0.3) exponential width for the option of
                    \ttt{mstc(67)=5}.
\itemt{parc(49) :} (D=0.2) ratio for exponential for  \ttt{mstc(67)=5}.
%-------------------
% string
%-------------------
\medskip

\itemt{parc(51) :} (D=2.0GeV/c$^2$) minimum mass of string
                                   for non-strange baryons.
\itemt{parc(52) :} (D=3.5GeV/c$^2$) mass cut off for non-strange baryon
                                    resonances.
\itemt{parc(53) :} (D=2.0GeV/$^2$) minimum mass of non-strange mesonic string.
\itemt{parc(54) :} (D=1.0GeV/fm) string tension.
\itemt{parc(55) :} (D=1.0fm/c) formation time.
\itemt{parc(56) :} (D=4.0GeV/c$^2$) invariant mass cut-off for the dipole
                     radiation of a string system below which soft gluon
                     radiations are terminated.

\itemt{parc(57) :} not used.
%not work
%\itemt{parc(57) :} (D=0.1) string life time parameter: 
%  parc()*mass, see mstc(68).

%-------------------
% soft interactions
%-------------------

\itemt{parc(61) :} (D=4.6 GeV) Parameter of $BB$ collision
   to use string model (GeV) except non-strange baryon-baryon collision,
   where resonance production cross sections are explicitly parametrized.
\itemt{parc(62) :} (D=2.8 GeV) Parameter of $MB$ collision
                                    to use string model.
\itemt{parc(63) :} (D=1.8 GeV) Parameter of $MM$ collision
                                  to use string model.
\itemt{parc(64) :} (D=1.09GeV/c$^2$) minimum value for the non-strange
                                     baryonic resonance states.
\itemt{parc(65) :} (D=0.7GeV/c$^2$) minimum value for the invariant mass
                  of the excited meson-like string system
                 in a hadron-hadron interaction.
\itemt{parc(66) :} (D=0.36GeV) width of the Gaussian $P_t$ distribution of
                   produced hadron in Lund string fragmentation 
                   (\ttt{parj(21)} in PYTHIA6.1).
\itemt{parc(67), parc(68) :} (D=0.1GeV/c, 1.4GeV/c)
     parameters in the distribution for the $P_t$ kick from soft interactions
      (\ttt{hipr1(19),hipr1(20)} in HIJING1).
\itemt{parc(69), parc(70) :} (D=0.05GeV/c, 1.0GeV/c)
     parameters in the distribution for the $P_t$ kick from resonance
    production interactions for the option \ttt{mstc(67)=6}.

%---------------
% hard
%---------------

\itemt{parc(71) :} (D=10GeV) lowest c.m. energy  to do pQCD scattering.
                     %(=parp(2))
\itemt{parc(72) :} (D=0.1) parameter for the mean reaction time for the
           parton-parton scattering: $\tau=\ttt{parc(71)}/Q^2$.
\itemt{parc(73) :} (D=0.1) parameter for the mean life time of time-like
           parton branching.
\itemt{parc(74) :} (D=1.0) parameter for the mean life time of space-like parton
           branching.
%-----------------------------
% cluster
%-----------------------------
\itemt{parc(151) :} (D=2.1fm) critical separation of two-baryons
             above which a cluster cannot be formed.
\itemt{parc(152) :} (D=0.3GeV/c$^2$) critical separation of two-baryons
                         in momentum space.

\itemt{parc(191) :} JAM version number.
\itemt{parc(192) :} JAM subversion number.
\itemt{parc(193) :} last year of modification for JAM.
\itemt{parc(194) :} last month of modification for JAM.
\itemt{parc(195) :} last day of modification for JAM.

\end{entry}

\bigskip
\bigskip
%%%%%%%%%%%%%%%%%%%%%%%%%%%%%%%%%%%%%%%%%%%%%%%%%%%%%%%%%%%%%%%%%%%%%%%%%%%%%
%     MSTD()
%%%%%%%%%%%%%%%%%%%%%%%%%%%%%%%%%%%%%%%%%%%%%%%%%%%%%%%%%%%%%%%%%%%%%%%%%%%%%

\begin{entry}
\itemt{mstd(1) :} projectile ID.
\itemt{mstd(2) :} projectile mass number.
\itemt{mstd(3) :} proton number in projectile.

\itemt{mstd(4) :} target ID.
\itemt{mstd(5) :} target mass number.
\itemt{mstd(6) :} proton number in target.

\itemt{mstd(11) :} number of all initial particles.
\itemt{mstd(12) :} initial total baryon number.
\itemt{mstd(13) :} initial total charge.
\itemt{mstd(14) :} initial total strangeness.
\itemt{mstd(15) :} number of cell for box simulation,
                         \ttt{mstd(15)}$=$\ttt{mstc(9)}.

\itemt{mstd(21) :} current event number.
\itemt{mstd(22) :} random seed of current event.
\itemt{mstd(23) :} current time step.
\itemt{mstd(24) :} number of event for each impact parameter bin.
\itemt{mstd(25) :} count of number of warnings.
\itemt{mstd(26) :} type of latest warning given.
\itemt{mstd(27) :} count of number of errors.
\itemt{mstd(28) :} type of latest errors given.
\itemt{mstd(29) :} collision counter(internal use).
\itemt{mstd(30) :} flag for absorbed particles.
  \begin{subentry}
     \itemt{$=1$}: there are absorbed (already dead) particles in JAM array.
     \itemt{$=0$}: No.
  \end{subentry}

%------------------
% Collision number
%------------------

\itemt{mstd(41) :} total number of elastic collisions in current event.
\itemt{mstd(42) :} total number of inelastic collisions in current event.
\itemt{mstd(43) :} total number of absorptions in current event.
\itemt{mstd(44) :} number of Baryon-Baryon collision in current event.
\itemt{mstd(45) :} number of Meson-Baryon collision in current event.
\itemt{mstd(46) :} number of Meson-Meson collision in current event.
\itemt{mstd(47) :} number of antibaryon-Baryon collision in current event.
\itemt{mstd(48) :} number of hadron-parton collisions in current event.
\itemt{mstd(49) :} number of parton-parton collisions in current event.
\itemt{mstd(50) :} number of decays in current event.
\itemt{mstd(51) :} number of Fermi blocks in current event.
\itemt{mstd(52) :} number of Low energy cuts in current event.
\itemt{mstd(53) :} number of decay after simulation. in current event.
\itemt{mstd(54) :} not used.
\comment{
mstd(54) : total number of forbidden collisions at energy conservation
           in current event (not used).
}
\itemt{mstd(55) :} total number of jet produced from hard scattering
                   in current event.
\itemt{mstd(56) :} counter for formation time of newly produced hadrons.
\itemt{mstd(57) :} counter for formation time of hadrons which contains original quarks.

\itemt{mstd(61) :} average number of total collision which produce strangeness.
\itemt{mstd(62) :} same as \ttt{mstd(61)}, but collision involved constituent. quarks.
\itemt{mstd(63) :} average number of strangeness production
                   from string fragmentation.
\itemt{mstd(64) :} average number of strangeness production 
                   from resonance decay.

\itemt{mstd(61-80) :} not used.



\itemt{mstd(81) :} Maximum number of \ttt{nv} actually used in simulation.
\itemt{mstd(82) :} Maximum number of \ttt{mentry} actually used.
\itemt{mstd(83) :} total number of quarks in current event.
\itemt{mstd(84) :} total number of gluons in current event.

\comment{
\itemt{mstd(91) :} number of JAMCLUST calls in present run.
\itemt{mstd(92) :} number of nuclear cluster in present run.
\itemt{mstd(99) :} max. number of outgoing flavor in jamrmas2.
}

\end{entry}

\bigskip
\bigskip
%=================================================================
% PARD()
%=================================================================
\begin{entry}
\itemt{pard(1) :} current global time.
\itemt{pard(2) :} current impact parameter(fm).
\itemt{pard(3) :} current minimum impact parameter.
\itemt{pard(4) :} impact parameter bin.
\itemt{pard(5) :} $\beta$ factor for the transformation to laboratory frame.
\itemt{pard(6) :} $\gamma$ factor for the transformation to laboratory frame.
\itemt{pard(7) :} $R_x$; x-coordinate shift of projectile.
\itemt{pard(8) :} $R_y$; y-coordinate shift of target.
\itemt{pard(9) :} initial total $p_x$.
\itemt{pard(10):} initial total $p_y$.
\itemt{pard(11):} initial total $p_z$.
\itemt{pard(12):} initial total energy.
\itemt{pard(13):} initial total energy per initial baryons.

\itemt{pard(14) :} laboratory beam energy in "GeV per nucleon".
\itemt{pard(15) :} laboratory  beam momentum in "GeV/c per nucleon".
\itemt{pard(16) :} colliding energy $\sqrt{s}$ per nucleon.
\itemt{pard(17) :} beam rapidity $y_p$.
\itemt{pard(18) :} c.m. momentum.

\medskip
%----------------
%Box parameters
%----------------
\itemt{pard(21) :}  cell length (fm).
\itemt{pard(22) :}  total energy/baryon (GeV) to set for box calculation.

\itemt{pard(31) :} $p_x$ of projectile.
\itemt{pard(32) :} $p_y$ of projectile.
\itemt{pard(33) :} $p_z$ of projectile.
\itemt{pard(34) :} mass of projectile.
\itemt{pard(35) :} $\beta$ of projectile.
\itemt{pard(36) :} $\gamma$ of projectile.
\itemt{pard(37) :} $r_x$ of projectile.
\itemt{pard(38) :} $r_y$ of projectile.
\itemt{pard(39) :} $r_z$ of projectile.
\itemt{pard(40) :} radius of projectile.

\itemt{pard(41) :} $p_x$ of target.
\itemt{pard(42) :} $p_y$ of target.
\itemt{pard(43) :} $p_z$ of target.
\itemt{pard(44) :} mass of target.
\itemt{pard(45) :} $\beta$ of target.
\itemt{pard(46) :} $\gamma$ of target.
\itemt{pard(47) :} $r_x$ of target.
\itemt{pard(48) :} $r_y$ of target.
\itemt{pard(49) :} $r_z$ of target.
\itemt{pard(50) :} radius of target.

%----------------
%Collisions 120-
%----------------

\itemt{pard(51) :} maximum c.m. distance  for $BB$ collision.
\itemt{pard(52) :} maximum c.m. distance  for $BB$ collision.
\itemt{pard(53) :} maximum c.m. distance  for $DN$ collision.
\itemt{pard(54) :} maximum c.m. distance  for $MB$ collision.
\itemt{pard(55) :} maximum c.m. distance  for $MM$ collision.
\itemt{pard(56) :} maximum c.m. distance  for $\Bbar B$ collision.


%See mstc(31)-mstc(38).
\itemt{pard(71) :} event average number of elastic collisions.
\itemt{pard(72) :} event average number of inelastic collisions.
\itemt{pard(73) :} event average number of absorptions.
\itemt{pard(74) :} event average number of $BB$ collisions.
\itemt{pard(75) :} event average number of $MB$ collisions.
\itemt{pard(76) :} event average number of $MM$ collisions.
\itemt{pard(77) :} event average number of $\Bbar B$ collisions.
\itemt{pard(78) :} event average number of decays.
\itemt{pard(79) :} event average number of Fermi blocks.
\itemt{pard(80) :} event average number of Low energy cuts.
\itemt{pard(81) :} event average number of decay after simulation.
\itemt{pard(82) :} event average multiplicity.
\itemt{pard(83) :} event average meson multiplicity.
\itemt{pard(84) :} not used.
\comment{
\itemt{pard(84) :} average number of forbidden collision
                       due to energy conservation.
}
\itemt{pard(85) :} average number of hadron-parton collisions.
\itemt{pard(86) :} average number of parton-parton collisions.
\itemt{pard(87) :} average number of hard scattering.
\itemt{pard(88) :} average formation time of newly produced hadrons.
\itemt{pard(89) :} average formation time of hadron which contains ordinal
                   constituent quark.

\itemt{pard(101)-pard(103) :}
        parameters of nuclear E.O.S. in BUU mode
     which are defined as
$$E_{pot}=\alpha {\rho_B\over\rho_0}
      + \beta \left({\rho_B \over \rho_0}\right)^\gamma ,$$
where $\alpha=\ttt{pard(101)}$,
      $\beta=\ttt{pard(102)}$,
and      $\gamma=\ttt{pard(103)}$.
Those values depend on the choice of compressibility
specified by \ttt{mstc(6)}.

\end{entry}


\bigskip

\begin{verbatim}
     common/jamdat2/pare(200),mste(200)
\end{verbatim}
{\bf Purpose:} to store some information to handle two-body collisions.
%-----------------------------------
% mste() Current two-body collision
%-----------------------------------
\begin{entry}

\itemt{mste(1) :} current collision channel.
 \begin{subentry}
    \itemt{$=0$ :} collision cannot happen.
    \itemt{$=1$ :} elastic collision.
    \itemt{$=2$ :} inelastic collisions.
    \itemt{$=3$ :} annihilations.
    \itemt{$=4$ :} string excitation.
    \itemt{$=5$ :} string excitation of nucleon-nucleon collisions.
    \itemt{$=6$ :} hard parton-parton scattering.
   \itemt{$=11$ :}  direct s-wave pion scattering.
   \itemt{$= -1$ :} Pauli-blocking.
   \itemt{$= -2$ :} error after the call of \ttt{jamcross}.
   \itemt{$=-9$ :}  hard scattering false.
   \itemt{$=-77$ :} soft interaction false.
   \itemt{$=-88$ :} energetically not allowed  collision.
   \itemt{$=-99$ :} something was wrong.
 \end{subentry}

\itemt{mste(2) :} collision type.
 \begin{subentry}
         \itemt{$=-1$ :} decay of resonance or string.
         \itemt{$=-2$ :} decay of resonance or string after simulation.
         \itemt{$=1$  :} baryon-baryon collision.
         \itemt{$=2$  :} meson-baryon collision.
         \itemt{$=3$  :} meson-meson collision.
         \itemt{$=4$  :} antibaryon-baryon collision.
         \itemt{$=5$  :} parton-hadron collision.
         \itemt{$=6$  :} parton-parton collision.
 \end{subentry}
\itemt{mste(3) :} outgoing channel for more than two-body final state.
 \begin{subentry}
         \itemt{$=0$ :} two-body final.
         \itemt{$=1$ :} $pp\to pp \pi$.
         \itemt{$=2$ :} $nn\to nn \pi$.
         \itemt{$=3$ :} $pn\to nn \pi$.
 \end{subentry}
\itemt{mste(4) :} status of the latest soft string excitation.
 \begin{subentry}
        \itemt{$=1$ :} double diffractive.
	\itemt{$=2$ :} projectile single diffractive.
	\itemt{$=3$ :} target single diffractive.
	\itemt{$=4$ :} non-diffractive.
 \end{subentry}

\itemt{mste(5) :} number of jet in least hard collision.
\itemt{mste(6) :} cell number \ttt{icoll(3,i)} in current collision.

\itemt{mste(21):} line number of the ingoing particle 1.
\itemt{mste(22):} KC code of the ingoing particle 1.
\itemt{mste(23):} line number of the ingoing particle 2.
\itemt{mste(24):} KC code of the ingoing particle 2.

\itemt{mste(40):} =1 during the final decay after simulation, otherwise =0.


\comment{
%ingoing particle 1
\medskip
\itemt{mste(21):} line number of the ingoing particle 1.
\itemt{mste(22):} KC code of the ingoing particle 1.
\itemt{mste(23):}41 line number of the ingoing particle 2.
\itemt{mste(24):}42 KC code the ingoing particle 2.
\itemt{mste(25):}61 line number  the outgoing particle 1.
\itemt{mste(26):}62 KC code of the outgoing particle 1.
\itemt{mste(27):}81 line number the outgoing particle 2.
\itemt{mste(28):}82 kc code the outgoing particle 2.
\itemt{mste(29):}101 line number of the outgoing particle 3.
\itemt{mste(30):}102 KC code of the outgoing particle 3.
\itemt{mste(31):}121 line number of the outgoing particle 4.
\itemt{mste(32):}122 kc code of the outgoing particle 4.

\itemt{mste(23):} \ttt{k(1,i)} of the ingoing particle 1.
\itemt{mste(24):} \ttt{k(2,i)} particle ID of the ingoing particle 1.
\itemt{mste(25):} \ttt{k(3,i)} charge of  ingoing particle 1.
\itemt{mste(26):} \ttt{k(4,i)} of the  ingoing particle 1.
\itemt{mste(27):} \ttt{k(5,i)} of the  ingoing particle 1.
\itemt{mste(28):} \ttt{k(6,i)} of the  ingoing particle 1.
\itemt{mste(29):} \ttt{k(7,i)} of the  ingoing particle 1.
\itemt{mste(30):} \ttt{k(8,i)} of the  ingoing particle 1.
\itemt{mste(31):} \ttt{k(9,i)} of the  ingoing particle 1.
\itemt{mste(32):} \ttt{k(10,i)} of the  ingoing particle 1.
\itemt{mste(33):} \ttt{k(11,i)} of the  ingoing particle 1.
\itemt{mste(34):} \ttt{kq(1,i)} of the  ingoing particle 1.
\itemt{mste(35):} \ttt{kq(2,i)} of the  ingoing particle 1.

\medskip
\itemt{mste(41)-mste(55):} same meaning as above for the ingoing particle 2.
\itemt{mste(61)-mste(75):} same meaning as above for the outgoing particle 1.
\itemt{mste(81)-mste(95):} same meaning as above for the outgoing particle 2.
\itemt{mste(101)-mste(115):} same meaning as above for the outgoing particle 3.
\itemt{mste(121)-mste(135):} same meaning as above for the outgoing particle 4.
}

\bigskip
\bigskip
%-------------------------
% PARE() Latest collision
%-------------------------
\itemt{pare(1) :} maximum time for collision search.
\itemt{pare(2) :} c.m.energy of two-body system in the latest collision.
\itemt{pare(3) :} $x\sigtot$ of latest collision, where $x$ denotes random
                  number.
\itemt{pare(4) :} total cross section of latest collision.
\itemt{pare(5) :} elastic cross section of latest collision.
\itemt{pare(6) :} impact parameter squared of latest collision.
\itemt{pare(7) :} c.m. momentum of latest collision.
\itemt{pare(8) :} collision time for particle 1.
\itemt{pare(9) :} collision time for particle 2.
\itemt{pare(11):} sum of energy of the two particles before collision
                   in the observer frame.

\itemt{pare(12):}$\beta_x$ for transformation from two-body c.m. frame to observer frame.
\itemt{pare(13):}$\beta_y$ for transformation from two-body c.m. frame to observer frame.
\itemt{pare(14):}$\beta_z$ for transformation from two-body c.m. frame to observer frame.
\itemt{pare(15):}$\gamma$ for transformation from two-body c.m. frame to observer frame.

\comment{
%ingoing particle 1
\itemt{pare(21):}\ttt{p(1,i)}$p_x$ of the ingoing particle 1 in the latest collision.
\itemt{pare(22):}\ttt{p(2,i)}$p_y$ of the ingoing particle 1 in the latest collision.
\itemt{pare(23):}\ttt{p(3,i)}$p_z$ of the ingoing particle 1 in the latest collision.
\itemt{pare(24):}\ttt{p(4,i)}$E$ of the ingoing particle 1 in the latest collision.
\itemt{pare(25):}\ttt{p(5,i)} mass of the ingoing particle 1 in the latest collision.
\itemt{pare(26):}\ttt{r(1,i)} of the ingoing particle 1 in the latest collision.
\itemt{pare(27):}\ttt{r(2,i)} of the ingoing particle 1 in the latest collision.
\itemt{pare(28):}\ttt{r(3,i)} of the ingoing particle 1 in the latest collision.
\itemt{pate(29):}\ttt{r(4,i)} of the ingoing particle 1 in the latest collision.
\itemt{pate(30):}\ttt{r(5,i)} of the ingoing particle 1 in the latest collision.
\itemt{pare(31):}\ttt{v(1,i)} of the ingoing particle 1 in the latest collision.
\itemt{pare(32):}\ttt{v(2,i)} of the ingoing particle 1 in the latest collision.
\itemt{pare(33):}\ttt{v(3,i)} of the ingoing particle 1 in the latest collision.
\itemt{pare(34):}\ttt{v(4,i)} of the ingoing particle 1 in the latest collision.
\itemt{pate(35):}\ttt{v(5,i)} of the ingoing particle 1 in the latest collision.
\itemt{pare(36):}\ttt{vq(1,i)} of the ingoing particle 1 in the latest collision.
\itemt{pare(37):}\ttt{vq(2,i)} of the ingoing particle 1 in the latest collision.
\itemt{pare(38):}\ttt{vq(3,i)} of the ingoing particle 1 in the latest collision.
\itemt{pare(39):}\ttt{vq(4,i)} of the ingoing particle 1 in the latest collision.
\itemt{pare(40):}\ttt{vq(5,i)} of the ingoing particle 1 in the latest collision.
\itemt{pare(41):}\ttt{vq(6,i)} of the ingoing particle 1 in the latest collision.
\itemt{pare(42):}\ttt{vq(7,i)} of the ingoing particle 1 in the latest collision.
\itemt{pare(43):}\ttt{vq(8,i)} of the ingoing particle 1 in the latest collision.
\itemt{pare(44):}\ttt{vq(9,i)} of the ingoing particle 1 in the latest collision.
\itemt{pare(45):}\ttt{vq(10,i)} of the ingoing particle 1 in the latest collision.
\medskip
%ingoing particle 2
\itemt{pare(46)-pare(70):} same meaning as above for the ingoing particle 2.

%outgoing particle 1
\itemt{pare(71)-pare(95) :} same meaning as above for the outgoing particle 1.
%outgoing particle 2
\itemt{pare(96)-pare(120) :} same meaning as above for the outgoing particle 2.
%outgoing particle 3
\itemt{pare(121)-pare(145) :} same meaning as above for the outgoing particle 3.
%outgoing particle 4
\itemt{pare(146)-pare(170) :} same meaning as above for the outgoing particle 4.
}

\end{entry}


\subsection{Other physics routines}

The routines which is important for the main task for the simulation
or is expected to be used frequently by the users
are listed here.

\begin{entry}
\itemt{subroutine jamread(nevent,bmin,bmax,dt,nstep,chfram,chbeam,chtarg,cwin)}

: to read input configuration file.

\itemt{subroutine jamlist(mlist)}

     : to list event record or particle data.

\itemt{subroutine jamname(kf1,kf2,kf3,chau)}

: to give the particle/parton/nucleus name as character string.
For nucleus,
$\ttt{kf1}=1000000000+1000000N_{n}+1000N_{p}+N_{\Lambda}$,

$\ttt{kf2}=1000000000+1000000N_{\Sigma^-}+1000N_{\Sigma^0}+N_{\Sigma^+}$,

$\ttt{kf3}=1000000000+1000000N_{\Xi^-}+1000N_{\Xi^0}+N_{\Omega^-}$.

where $N_{n},N_{p},N_{\Lambda},N_{\Sigma^-},N_{\Sigma^0},N_{\Sigma^+},
N_{\Xi^-},N_{\Xi^0},N_{\Omega^-}$ represent the number of particle
contained in nucleus respectively.
Otherwise, \ttt{kf1} is usual FK flavor code and \ttt{kf2} and \ttt{kf3}
are not used.

\itemt{subroutine jamboost}

: to boost the ground state nuclei/particles.

\itemt{subroutine jamgrund}

: to make the ground state of target and projectile.

\itemt{subroutine jamjeti}

: to initialize the PYTHIA and HIJING routines to simulate 
 string fragmentation, jet production and calculation of jet-cross sections.

\itemt{subroutine jamupdat(mupda,lfn)}

 : to facilitates the updating of particle and decay data
    by allowing it to be done in an external file.

\itemt{subroutine jamcoll}

: to administer the two-body collisions and decays.

\itemt{subroutine jamcross}
: to administer the hadron-hadron collisions.

\itemt{subroutine jamscatt}: to perform two-body collisions.
\itemt{subroutine jamabsrb}: to treat annihilation scattering.
\itemt{subroutine jamchanl}: to select final scattering channel.
\itemt{subroutine jamangel}: to calculate elastic angular distribution.
\itemt{function jamslope}:   to give slope parameter in two-body collisions.
\itemt{subroutine jamangin}: to calculate inelastic angular distribution.
\itemt{subroutine jamsoft}:  to give masses of the excited nucleons.
\itemt{subroutine jamhard}:  to perform jets production.
\itemt{subroutine jamdec}:   to administrate the fragmentation of jet system or decay.
\itemt{subroutine jamjdec}:  to fragment jet system by Lund model.
\itemt{subroutine jamrdec}:  to calculate decay of hadrons.


\end{entry}


\section{Instructions on how to run the program}

 Program usage of JAM is very similar to other famous event generators;
 PYTHIA~\cite{pythia},FRITIOF~\cite{fritiof},HIJING~\cite{hijing}
 and VNI~\cite{vni}. for example, (1) main program by users,
 (2) Particle flavor code is the same as particle data group particle
     numbering scheme.

\begin{itemize}
\item Write main program in which at least 
       \ttt{call jaminit()}, \ttt{call jamevt(iev)} and
        \ttt{call jamfin} must be written.
\item Make executable  file after rewriting Makefile for suitable
      machine and your main program name.
\end{itemize}

\subsection{Output from JAM}
Some output files are generated in JAM.
\begin{itemize}
\item General information on inputs and outputs 
        are written in file
     specified by \ttt{fname(2)} (default name can be seen in \ttt{jamdata}).
\item
    Some error or warning messages are written in file 
    \ttt{fname(3)}.
\item
    Current event number generated are written in file \ttt{fname(3)},
    if $\ttt{mstc(39)}=1$.
\item
   If switches for event study are on,
   files are created, for example,  with the file name \ttt{JAM162.DAT}
   in the case of $\ttt{mstc(162)}=1$.

\end{itemize}



\subsection{Most simple example: Main program and its output}
Here is an most simple example main program. This simulates
S(200AGeV/c)+S collision for impact parameter sampled by $b<1$fm.

\begin{verbatim}
c...A main program for S(200GeV/c)+S

      include 'jam1.inc'
      include 'jam2.inc'
      character frame*8,proj*8,targ*8,cwin*15

c....Initialize JAM
      mstc(1)=48827       ! random seed.
      mevent=1            ! total simulation event
      bmin=0.0d0          ! minimum impact parameter
      bmax=-1.0d0         ! maximum impact parameter
      dt=100.0d0          ! collision time(fm/c)
      nstep=1             ! time step
      cwin='200gevc'      ! incident energy
      frame='nn'          ! comp. frame
      proj='32S'          ! projectile
      targ='32S'          ! target
      mstc(156)=1         ! analysis of collision distribution on.

c...Initialize JAM.
      call jaminit(mevent,bmin,bmax,dt,nstep,frame,proj,targ,cwin)

c...Simulation start.
      do iev=1,mevent

c...Simulate one event.
        call jamevt(iev)

c...List phase space data.
        call jamlist(1)

      end do

c...Final output.
      call jamfin

      end
\end{verbatim}


Some global information is written in the file "\ttt{JAMRUN.DAT}". 

\begin{verbatim}
 ******************************************************************************
 ******************************************************************************
 **                                                                          **
 **                                                                          **
 **   Welcome to the JAM Monte Carlo!                                        **
 **                                                                          **
 **   JJJJJJJJ  AAAA      MM     MMM                                         **
 **      JJ    AA  AA     MMM    MMM                                         **
 **      JJ   AAAAAAAA    MM MMMM MM                                         **
 **   J  JJ  AA      AA   MM      MM                                         **
 **   JJJJJ AA        AA  MM      MM                                         **
 **                                                                          **
 **    This is JAM  version 1.009                                            **
 **  Last date of change:  6 Sep 1999                                        **
 **                                                                          **
 **                                                                          **
 **                                                                          **
 **  JAM is a free software which can                                        **
 **  be distributed under the GNU Gen                                        **
 **  eral Public License. There is no                                        **
 **  warranty for the program.                                               **
 **                                                                          **
 **  Copyright (c) 1998 Yasushi Nara                                         **
 **                                                                          **
 **  Main author: Yasushi Nara;                                              **
 **  Advance Science Research Center, Japan Atomic Energy Research Institute **
 **  Tokai-mura, Naka-gun, Ibaraki-ken  319-1195, Japan                      **
 **  e-mail: ynara@bnl.gov                                                   **
 **                                                                          **
 **                                                                          **
 **                                                                          **
 ******************************************************************************
 ******************************************************************************


      *******************************************************************
      *                                                                 *
      *          Reaction:                                              *
      *                                                                 *
      *              mass  32( 16, 16) ==> mass  32( 16, 16)            *
      *                                                                 *
      *              Collision of 32S     on 32S                        *
      *                                                                 *
      *                                                                 *
      *               Beam energy   =    199.06 A GeV                   *
      *               Beam momentum =    200.00 A GeV/c                 *
      *               C.M. energy   =     19.43 A GeV                   *
      *                                                                 *
      *                               Beam        Target                *
      *          Velocity/c:       0.9953164     -.9953164              *
      *          Gamma factor:      10.344        10.344                *
      *          p_x(GeV/c):         0.000         0.000                *
      *          p_z(GeV/c):         9.667        -9.667                *
      *          r_x(fm):            0.000         0.000                *
      *          r_z(fm):           -0.816         0.816                *
      *                                                                 *
      *        Impact parameters are distributed according to b^2:      *
      *                  0.000 < b <   1.000 (fm)                       *
      *                                                                 *
      *******************************************************************



      * Input output file name: "            non"
      * Computational frame: nucl.-nucl. c.m.
      * Random seed     =     48827
      * # of event      =      1
      * Time step size  =100.000 fm/c
      * # of timestep   =      1




                           per simulation      1 runs       1 parallel runs
      -----------------------------------
       Total collisions      =   478.0000
       Elastic collisions    =    87.0000
       Inelastic collisions  =   391.0000
       Absorptions           =   213.0000
       baryon-baryon coll.   =    60.0000
       meson-baryon coll.    =   191.0000
       meson-meson coll.     =   222.0000
       anti-B-B coll.        =     5.0000
       hadron-parton coll.   =     0.0000
       parton-parton coll.   =     0.0000
       Resonance decays      =   405.0000
       Total particles       =   390.0000
       Mesons                =   316.0000
       Pauli blocks          =     1.0000
       Blocks due to E.con.  =     0.0000
       Low energy cuts       =  2211.    
       Decay after simul.    =     7.0000
       Mean formation time          =     0.9814  114.0000
       Mean const. formation time   =     1.4099  184.0000


       average number of jet=    0.0000
       max. number of nv=      607
       max. number of mentry=      270


 ==============================================================================
          FINAL HADRON MULTIPLICITIES
 ==============================================================================

 total particle   390.000000000000     
 charged particle   214.000000000000     
 negative particle   91.0000000000000     
 positive particle   91.0000000000000     
   22        22  gamma                 1.00    
  132       111  pi0                   115.    
  133       211  pi+                   87.0      pi-                   84.0    
  153       221  eta                   6.00    
  173       311  K0                    10.0      Kbar0                 7.00    
  174       321  K+                    5.00      K-                    3.00    
  252      2112  n0                    35.0      nbar0                 1.00    
  253      2212  p+                    28.0      pbar-                 3.00    
  327      3112  Sigma-                1.00      Sigmabar+            0.000E+00
  328      3212  Sigma0                1.00      Sigmabar0            0.000E+00
  329      3222  Sigma+                3.00      Sigmabar-            0.000E+00

      * CPU time=   0 h   0 m  21 s
      *******************************
\end{verbatim}

\subsection{Input from a input configuration file}
As a default \ttt{fname(1)} is set to be '0'.
In this case, JAM does not use any input file.
If there is a statement in a main program like
\begin{verbatim}
      fname(1)='jam.cfg'
\end{verbatim}
JAM read input parameter from file name \ttt{'jam.cfg'}.
\begin{itemize}
\item \ttt{event, bmin, bmax, projectile, target, timestep, dt, win}
      are used to specify simulation conditions.
\item '\#' and '!' can be used to comment.  Blank line is also allowed.
\item Both upper and lower case can be used.
\item If you want to change default value of \ttt{mstc()}, \ttt{parc()},
        put, for example, \ttt{mstc(65)= 2}.
\item Input file should be ended by \ttt{'end'} command.
\end{itemize}

Here is a example input file.

\begin{verbatim}
event=100             # Total number of siumulation run
MSTC(1) =8595934      # random seed.

proj =208Pb        # projectile
targ =208Pb        # target
win= 158gev        # incident energy
bmin =  0.0        # minimum impact parameter in fm.
bmax = -3.2        # maximam impact parameter in fm with $b^2$ distribution.

frame= nn         # nn c.m. frame.
dt= 100.0         # Time step size(fm/c)
timestep = 1      # total number of time step.

#***************************************************>
#***** Optional input of mstc and parc   ***********>
#***************************************************>
fname(2)=XXX     # file name for simulation conditions.
fname(3)=YYY     # file name for Some Information

Mstc(8)=1        # job mode. =0:Job.
mstc(5)= 1       # impact parameter bin number
parc(6)= 10.0    # scale of display
mstc(16)= 0      # display on/off.

# option for analysis
mstc(155)=1     # calculate transverse flow.
mstc(156)=1     # energy distribution of collisions on.
mstc(162)=0     # Output collision histroy off.

mstc(163)=1     ! time evolution of directed transverse flow
mstc(164)=0     ! Output phase space data in time interval
mstc(165)=0     ! Output time evolution of particle yield
mstc(166)=0     ! Output time evolution of particle density
parc(7)= 2.0    ! Output time interval (fm/c)

end             ! Input end
\end{verbatim}


%\section{Modification and further development}

{\Large \bf Acknowledgments}\\[1ex]

I am very grateful to K.K. Geiger for useful discussions.
I would like to thank A. Ohnishi
  for continuing encouragement and discussions.
I acknowledge the inclusion of matter calculation part
 to Toshiki Maruyama.
I also wish to acknowledge careful reading of the manuscript
 to S. Chiba and A. Ohnishi.
I'd also like to thank Brookhaven National Laboratory for giving the
author an opportunity to stay.


\section*{Appendix}

\subsection*{A: Particle codes}\label{app:particle}

  Here the particle flavor codes defined in JAM are listed.
  Corresponding antiparticle has the negative of its flavor code.
  I refer the reader to Ref.~\cite{PDG96} for the detailed meaning of
   the particle numbering scheme.
The particle name printed in the table is obtained using subroutine
 \ttt{jamname}.

\begin{table}[ptb]
\captive{Quark, lepton , gauge boson and diquark codes.
\protect\label{t:codeone} }  \\
\vspace{1ex}
\begin{center}
\begin{tabular}{|c|c|c||c|c|c|@{\protect\rule{0mm}{\tablinsep}}}
\hline
KF & Name & Printed & KF & Name & Printed \\
\hline
    1 & $\d$ & \ttt{d}   &  11 & $\e^-$ &  \ttt{e-}    \\
    2 & $\u$ & \ttt{u}   &  12 & $\nu_{\e}$ &  \ttt{nu\_e}   \\
    3 & $\s$ & \ttt{s}   &  13 & $\mu^-$ &  \ttt{mu-}   \\
    4 & $\c$ & \ttt{c}   &  14 & $\nu_{\mu}$ & \ttt{nu\_mu}   \\
    5 & $\b$ & \ttt{b}   &  15 & $\tau^-$ & \ttt{tau-}   \\
    6 & $\t$ & \ttt{t}   &  16 & $\nu_{\tau}$ &  \ttt{nu\_tau}   \\
    7 & $\lrm$ & \ttt{l}   &  17 & $\chi^-$ & \ttt{chi-}  \\
    8 & $\hrm$ & \ttt{h}   &  18 & $\nu_{\chi}$ & \ttt{nu\_chi}  \\
   21 & $\g$ & \ttt{g}             &    & &   \\
   22 & $\gamma$ & \ttt{gamma}     & 32 & $\Z'^0$ & \ttt{Z'0}   \\
   23 & $\Z^0$ & \ttt{Z0}          & 33 & $\Z''^0$ & \ttt{Z"0}   \\
   24 & $\W^+$ & \ttt{W+}          & 34 & $\W'^+$ & \ttt{W'+}   \\
   25 & $\H^0$ & \ttt{H0}          & 35 & $\H'^0$ & \ttt{H'0}   \\
   26 & &                          & 36 & $\A^0$ & \ttt{A0}   \\
   27 & &                          & 37 & $\H^+$ & \ttt{H+}   \\
   91 &  cluster & \ttt{cluster}   &    & &\\
   92 & string & \ttt{string}      &    & &\\
      &          &             & 1103 & $\d\d_1$ & \ttt{dd\_1}  \\
 2101 & $\u\d_0$ & \ttt{ud\_0} & 2103 & $\u\d_1$ & \ttt{ud\_1}  \\
      &          &             & 2203 & $\u\u_1$ & \ttt{uu\_1}  \\
 3101 & $\s\d_0$ & \ttt{sd\_0} & 3103 & $\s\d_1$ & \ttt{sd\_1}  \\
 3201 & $\s\u_0$ & \ttt{su\_0} & 3203 & $\s\u_1$ & \ttt{su\_1}  \\
      &          &             & 3303 & $\s\s_1$ & \ttt{ss\_1}  \\
\hline
\end{tabular}
\end{center}
\end{table}
 
%
%   MESONS
%
\begin{table}[ptb]
\captive{light and strange meson codes\protect\label{t:codetwo} }
\vspace{1ex}
\begin{center}
\begin{tabular}{|c|c|c||c|c|c||c|c|c|@{\protect\rule{0mm}{\tablinsep}}} \hline
KF & Name & Printed & KF & Name & Printed & KF & Name & Printed \\ \hline
      111 & $\pi^0$           &  \ttt{pi0      }   &     10220 & $\sigma$          &  \ttt{sigma-m     }   &       130 & K$_{L}^{0}$       &  \ttt{K\_L0     }  \\
      211 & $\pi^+$           &  \ttt{pi+      }   &       221 & $\eta$            &  \ttt{eta         }   &       310 & K$_{S}^{0}$       &  \ttt{K\_S0     }  \\
      113 & $\rho^0$          &  \ttt{rho0     }   &       223 & $\omega$          &  \ttt{omega       }   &       311 & K$^{0}$           &  \ttt{K0        }  \\
      213 & $\rho^+$          &  \ttt{rho+     }   &       225 & f$_2$             &  \ttt{f\_2        }   &       321 & K$^{+}$           &  \ttt{K+        }  \\
    10111 & a$_0^0$           &  \ttt{a\_00    }   &       331 & $\eta'$           &  \ttt{eta'        }   &       313 & K$^{*0}$          &  \ttt{K*0       }  \\
    10211 & a$_0^+$           &  \ttt{a\_0+    }   &       333 & $\phi$            &  \ttt{phi         }   &       323 & K$^{*+}$          &  \ttt{K*+       }  \\
    20113 & a$_1^0$           &  \ttt{a\_10    }   &       335 & f$'_2$            &  \ttt{f'\_2       }   &     10313 & K$_1^{0}$         &  \ttt{K\_10     }  \\
    20213 & a$_1^+$           &  \ttt{a\_1+    }   &     10221 & f$_0$             &  \ttt{f\_0        }   &     10323 & K$_1^{+}$         &  \ttt{K\_1+     }  \\
      115 & a$_2^0$           &  \ttt{a\_20    }   &     10223 & h$_1$             &  \ttt{h\_1        }   &       315 & K$^{*0}_2$        &  \ttt{K*\_20    }  \\
      215 & a$_2^+$           &  \ttt{a\_2+    }   &     10331 & f$'_0$            &  \ttt{f'\_0       }   &       325 & K$^{*+}_2$        &  \ttt{K*\_2+    }  \\
    10113 & b$_1^0$           &  \ttt{b\_10    }   &     10333 & h$'_1$            &  \ttt{h'\_1       }   &     10311 & K$^{*0}_0$        &  \ttt{K*\_00    }  \\
    10213 & b$_1^+$           &  \ttt{b\_1+    }   &     20221 & $\eta(1295)$      &  \ttt{eta(1295)   }   &     10321 & K$^{*+}_0$        &  \ttt{K*\_0+    }  \\
    20111 & $\pi(1300)^0$     &  \ttt{pi(1300)0}   &     20223 & f$_1$             &  \ttt{f\_1        }   &     20313 & K$^{*0}_1$        &  \ttt{K*\_10    }  \\
    20211 & $\pi(1300)^+$     &  \ttt{pi(1300)+}   &     20333 & f$'_1$            &  \ttt{f'\_1       }   &     20323 & K$^{*+}_1$        &  \ttt{K*\_1+    }  \\
    10115 & $\pi_2(1670)^0$   &  \ttt{pi\_2(1670)0}   &     30221 & f$_0(1300)$       &  \ttt{f\_0(1300)  }   &     30313 & K(1410)$^{0}$     &  \ttt{K*(1410)0 }  \\
    10215 & $\pi_2(1670)^+$   &  \ttt{pi\_2(1670)+}   &     30223 & $\omega(1420)$    &  \ttt{omega(1420) }   &     30323 & K(1410)$^{+}$     &  \ttt{K*(1410)+ }  \\
    30113 & $\rho(1465)^0$    &  \ttt{rho(1465)0}   &     30333 & $\phi(1680)$      &  \ttt{phi(1680)   }   &     40313 & K(1680)$^{0}$     &  \ttt{K*(1680)0 }  \\
    30213 & $\rho(1465)^+$    &  \ttt{rho(1465)+}   &     50223 & f$_1(1510)$       &  \ttt{f\_1(1510)  }   &     40323 & K(1680)$^{+}$     &  \ttt{K*(1680)+ }  \\
    40113 & $\rho(1700)^0$    &  \ttt{rho(1700)0}   &     60223 & $\omega(1600)$    &  \ttt{omega(1600) }   &       317 & K$_3(1780)^{0}$   &  \ttt{K\_3(1780)0}  \\
    40213 & $\rho(1700)^+$    &  \ttt{rho(1700)+}   &           &               &                           &       327 & K$_3(1780)^{+}$   &  \ttt{K\_3(1780)+}  \\
          &               &                        &           &               &                           &     10315 & K$_2(1770)^{0}$   &  \ttt{K\_2(1770)0}  \\
          &               &                        &           &               &                           &     10325 & K$_2(1770)^{+}$   &  \ttt{K\_2(1770)+}  \\
          &               &                        &           &               &                           &     20315 & K$_2(1820)^{0}$   &  \ttt{K\_2(1820)0}  \\
          &               &                        &           &               &                           &     20325 & K$_2(1820)^{+}$   &  \ttt{K\_2(1820)+}  \\
\hline
\end{tabular}
\end{center}
\end{table}

%
%  Non-Strange Baryons
%

\begin{table}[ptb] 
\captive{ Non-strange baryon codes\protect\label{t:codethree} }  
\vspace{1ex} 
\begin{center} 
\begin{tabular}{|c|c|c||c|c|c| @{\protect\rule{0mm}{\tablinsep}}}   \hline
KF & Name & Printed & KF & Name & Printed\\ \hline
2112 & n & \ttt{n0}                           & 1114 & $\Delta^-$  & \ttt{delta-}                  \\
2212 & p & \ttt{p+}                           & 2114 & $\Delta^0$  & \ttt{delta0}                  \\
    12112 & $N(1440)^0$ &  \ttt{N(1440)0}     & 2214 & $\Delta^+$  & \ttt{delta+}                  \\
    12212 & $N(1440)^+$ &  \ttt{N(1440)+}     & 2224 & $\Delta^{++}$ & \ttt{delta++}                 \\
     1214 & $N(1520)^0$ &  \ttt{N(1520)0}     &     31114 & $\Delta(1600)^{- }$ &  \ttt{D(1600)- } \\
     2124 & $N(1520)^+$ &  \ttt{N(1520)+}     &     32114 & $\Delta(1600)^{0 }$ &  \ttt{D(1600)0 } \\
    22112 & $N(1535)^0$ &  \ttt{N(1535)0}     &     32214 & $\Delta(1600)^{+ }$ &  \ttt{D(1600)+ } \\
    22212 & $N(1535)^+$ &  \ttt{N(1535)+}     &     32224 & $\Delta(1600)^{++}$ &  \ttt{D(1600)++} \\
    32112 & $N(1650)^0$ &  \ttt{N(1650)0}     &      1112 & $\Delta(1620)^{- }$ &  \ttt{D(1620)- } \\
    32212 & $N(1650)^+$ &  \ttt{N(1650)+}     &      1212 & $\Delta(1620)^{0 }$ &  \ttt{D(1620)0 } \\
     2116 & $N(1675)^0$ &  \ttt{N(1675)0}     &      2122 & $\Delta(1620)^{+ }$ &  \ttt{D(1620)+ } \\
     2216 & $N(1675)^+$ &  \ttt{N(1675)+}     &      2222 & $\Delta(1620)^{++}$ &  \ttt{D(1620)++} \\
    12116 & $N(1680)^0$ &  \ttt{N(1680)0}     &     11114 & $\Delta(1700)^{- }$ &  \ttt{D(1700)- } \\
    12216 & $N(1680)^+$ &  \ttt{N(1680)+}     &     12114 & $\Delta(1700)^{0 }$ &  \ttt{D(1700)0 } \\
    21214 & $N(1700)^0$ &  \ttt{N(1700)0}     &     12214 & $\Delta(1700)^{+ }$ &  \ttt{D(1700)+ } \\
    22124 & $N(1700)^+$ &  \ttt{N(1700)+}     &     12224 & $\Delta(1700)^{++}$ &  \ttt{D(1700)++} \\
    42112 & $N(1710)^0$ &  \ttt{N(1710)0}     &     11112 & $\Delta(1900)^{- }$ &  \ttt{D(1900)- } \\
    42212 & $N(1710)^+$ &  \ttt{N(1710)+}     &     11212 & $\Delta(1900)^{0 }$ &  \ttt{D(1900)0 } \\
    31214 & $N(1720)^0$ &  \ttt{N(1720)0}     &     12122 & $\Delta(1900)^{+ }$ &  \ttt{D(1900)+ } \\
    32124 & $N(1720)^+$ &  \ttt{N(1720)+}     &     12222 & $\Delta(1900)^{++}$ &  \ttt{D(1900)++} \\
    11218 & $N(1990)^0$ &  \ttt{N(1990)0}     &      1116 & $\Delta(1905)^{- }$ &  \ttt{D(1905)- } \\
    12128 & $N(1990)^+$ &  \ttt{N(1990)+}     &      1216 & $\Delta(1905)^{0 }$ &  \ttt{D(1905)0 } \\
          &              &                    &      2126 & $\Delta(1905)^{+ }$ &  \ttt{D(1905)+ } \\
          &              &                    &      2226 & $\Delta(1905)^{++}$ &  \ttt{D(1905)++} \\
          &              &                    &     21112 & $\Delta(1910)^{- }$ &  \ttt{D(1910)- } \\
          &              &                    &     21212 & $\Delta(1910)^{0 }$ &  \ttt{D(1910)0 } \\
          &              &                    &     22122 & $\Delta(1910)^{+ }$ &  \ttt{D(1910)+ } \\
          &              &                    &     22222 & $\Delta(1910)^{++}$ &  \ttt{D(1910)++} \\
          &              &                    &     21114 & $\Delta(1920)^{- }$ &  \ttt{D(1920)- } \\
          &              &                    &     22114 & $\Delta(1920)^{0 }$ &  \ttt{D(1920)0 } \\
          &              &                    &     22214 & $\Delta(1920)^{+ }$ &  \ttt{D(1920)+ } \\
          &              &                    &     22224 & $\Delta(1920)^{++}$ &  \ttt{D(1920)++} \\
          &              &                    &     11116 & $\Delta(1930)^{- }$ &  \ttt{D(1930)- } \\
          &              &                    &     11216 & $\Delta(1930)^{0 }$ &  \ttt{D(1930)0 } \\
          &              &                    &     12126 & $\Delta(1930)^{+ }$ &  \ttt{D(1930)+ } \\
          &              &                    &     12226 & $\Delta(1930)^{++}$ &  \ttt{D(1930)++} \\
          &              &                    &      1118 & $\Delta(1950)^{- }$ &  \ttt{D(1950)- } \\
          &              &                    &      2118 & $\Delta(1950)^{0 }$ &  \ttt{D(1950)0 } \\
          &              &                    &      2218 & $\Delta(1950)^{+ }$ &  \ttt{D(1950)+ } \\
          &              &                    &      2228 & $\Delta(1950)^{++}$ &  \ttt{D(1950)++} \\
\hline
\end{tabular} 
\end{center} 
\end{table} 

%
%  Strange Baryons
%

\begin{table}[ptb] 
\captive{Strange baryon codes\protect\label{t:codefoure} }  
\vspace{1ex} 
\begin{center} 
\begin{tabular}{|c|c|c||c|c|c| @{\protect\rule{0mm}{\tablinsep}}}   \hline
KF & Name & Printed & KF & Name & Printed\\ \hline
3122 & $\Lambda^0$ & \ttt{Lambda0}                 & 3112 & $\Sigma^-$     & \ttt{Sigma-}               \\
    13122 & $\Lambda(1405)^0$ &  \ttt{L(1405)0}    & 3212 & $\Sigma^0$     & \ttt{Sigma0}               \\
     3124 & $\Lambda(1520)^0$ &  \ttt{L(1520)0}    & 3222 & $\Sigma^+$     & \ttt{Sigma+}               \\
    23122 & $\Lambda(1600)^0$ &  \ttt{L(1600)0}    & 3114 & $\Sigma^{*-}$  & \ttt{Sigma*-}              \\
    33122 & $\Lambda(1670)^0$ &  \ttt{L(1670)0}    & 3214 & $\Sigma^{*0}$  & \ttt{Sigma*0}              \\
    13124 & $\Lambda(1690)^0$ &  \ttt{L(1690)0}    & 3224 & $\Sigma^{*+}$  & \ttt{Sigma*+}              \\
    43122 & $\Lambda(1800)^0$ &  \ttt{L(1800)0}    &     13112 & $\Sigma(1660)^-$ &  \ttt{S(1660)- }    \\
    53122 & $\Lambda(1810)^0$ &  \ttt{L(1810)0}    &     13212 & $\Sigma(1660)^0$ &  \ttt{S(1660)0 }    \\
     3126 & $\Lambda(1820)^0$ &  \ttt{L(1820)0}    &     13222 & $\Sigma(1660)^+$ &  \ttt{S(1660)+ }    \\
    13126 & $\Lambda(1830)^0$ &  \ttt{L(1830)0}    &     13114 & $\Sigma(1670)^-$ &  \ttt{S(1670)- }    \\
    23124 & $\Lambda(1890)^0$ &  \ttt{L(1890)0}    &     13214 & $\Sigma(1670)^0$ &  \ttt{S(1670)0 }    \\
     3128 & $\Lambda(2100)^0$ &  \ttt{L(2100)0}    &     13224 & $\Sigma(1670)^+$ &  \ttt{S(1670)+ }    \\
    23126 & $\Lambda(2110)^0$ &  \ttt{L(2110)0}    &     23112 & $\Sigma(1750)^-$ &  \ttt{S(1750)- }    \\
          &              &                         &     23212 & $\Sigma(1750)^0$ &  \ttt{S(1750)0 }    \\
3312 & $\Xi^-$     & \ttt{Xi-}                     &     23222 & $\Sigma(1750)^+$ &  \ttt{S(1750)+ }    \\
3322 & $\Xi^0$     & \ttt{Xi0}                     &      3116 & $\Sigma(1775)^-$ &  \ttt{S(1775)- }    \\
3312 & $\Xi^{*-}$  & \ttt{Xi*-}                    &      3216 & $\Sigma(1775)^0$ &  \ttt{S(1775)0 }    \\
3314 & $\Xi^{*0}$  & \ttt{Xi*0}                    &      3226 & $\Sigma(1775)^+$ &  \ttt{S(1775)+ }    \\
    13312 & $\Xi(1690)^-$ &  \ttt{X(1690)- }       &     13116 & $\Sigma(1915)^-$ &  \ttt{S(1915)- }    \\
    13322 & $\Xi(1690)^0$ &  \ttt{X(1690)0 }       &     13216 & $\Sigma(1915)^0$ &  \ttt{S(1915)0 }    \\
    13314 & $\Xi(1820)^-$ &  \ttt{X(1820)- }       &     13226 & $\Sigma(1915)^+$ &  \ttt{S(1915)+ }    \\
    13324 & $\Xi(1820)^0$ &  \ttt{X(1820)0 }       &     23114 & $\Sigma(1940)^-$ &  \ttt{S(1940)- }    \\
    23312 & $\Xi(1950)^-$ &  \ttt{X(1950)- }       &     23214 & $\Sigma(1940)^0$ &  \ttt{S(1940)0 }    \\
    23322 & $\Xi(1950)^0$ &  \ttt{X(1950)0 }       &     23224 & $\Sigma(1940)^+$ &  \ttt{S(1940)+ }    \\
     3316 & $\Xi(2030)^-$ &  \ttt{X(2030)- }       &      3118 & $\Sigma(2030)^-$ &  \ttt{S(2030)- }    \\
     3326 & $\Xi(2030)^0$ &  \ttt{X(2030)0 }       &      3218 & $\Sigma(2030)^0$ &  \ttt{S(2030)0 }    \\
          &              &                         &      3228 & $\Sigma(2030)^+$ &  \ttt{S(2030)+ }    \\
3334 & $\Omega^-$  & \ttt{Omega-}                  &           &              &                         \\
\hline
\end{tabular} 
\end{center} 
\end{table} 

%
% Heavy particles.
%
\begin{table}[ptb]
\captive{Heavy meson codes\protect\label{t:codefive} }
\vspace{1ex}
\begin{center}
\begin{tabular}{|c|c|c||c|c|c|@{\protect\rule{0mm}{\tablinsep}}} \hline
KF & Name & Printed & KF & Name & Printed \\ \hline
   411 & $D^+$        & \ttt{D+}       &    511 & $B^0$        & \ttt{B0}       \\
   413 & $D^{*+}$     & \ttt{D*+}      &    513 & $B^{*0}$     & \ttt{B*0}      \\
   415 & $D^{*+}_2$   & \ttt{D*\_2+}   &    515 & $B^{*0}_2$   & \ttt{B*\_20}   \\
   421 & $D^0$        & \ttt{D0}       &    521 & $B^+$        & \ttt{B+}       \\
   423 & $D^{*0}$     & \ttt{D*0}      &    523 & $B^{*+}$     & \ttt{B*+}      \\
   425 & $D^{*0}_2$   & \ttt{D*\_20}   &    525 & $B^{*+}_2$   & \ttt{B*\_2+}   \\
   431 & $D_s^+$      & \ttt{D\_s+}    &    531 & $B_s^0$      & \ttt{B\_s0}    \\
   433 & $D^{*+}_s$   & \ttt{D*\_s+}   &    533 & $B^{*0}_s$   & \ttt{B*\_s0}   \\
   435 & $D^{*+}_{2s}$& \ttt{D*\_2s+}  &    535 & $B^{*0}_{2s}$& \ttt{B*\_2s0}  \\
   441 & $\eta_c$     & \ttt{eta\_c}   &    541 & $B_c^+$      & \ttt{B\_c+}    \\
   443 & $J/\psi$     & \ttt{J/psi}    &    543 & $B^{*+}_c$   & \ttt{B*\_c+}   \\
   445 & $\chi_{2c}$  & \ttt{chi\_2c}  &    545 & $B^{*+}_{2c}$& \ttt{B*\_2c+}  \\
 10411 & $D^{*+}_0$   & \ttt{D*\_0+}   &    551 & $\eta_b$     & \ttt{eta\_b}   \\
 10413 & $D_1^+$      & \ttt{D\_1+}    &    553 & $\Upsilon$   & \ttt{Upsilon}  \\
 10421 & $D^{*0}_0$   & \ttt{D*\_00}   &    555 & $\chi_{2b}$  & \ttt{chi\_2b}  \\
 10423 & $D_1^0$      & \ttt{D\_10}    &  10511 & $B^{*0}_0$   & \ttt{B*\_00}   \\
 10431 & $D^{*+}_{0s}$& \ttt{D*\_0s+}  &  10513 & $B_1^0$      & \ttt{B\_10}    \\
 10433 & $D_{1s}^+$   & \ttt{D\_1s+}   &  10521 & $B^{*+}_0$   & \ttt{B*\_0+}   \\
 10441 & $\chi_0^c$   & \ttt{chi\_0c}  &  10523 & $B_1^+$      & \ttt{B\_1+}    \\
 10443 & $h_{1c}$     & \ttt{h\_1c}    &  10531 & $B^{*0}_{0s}$& \ttt{B*\_0s0}  \\
 20413 & $D^{*+}_1$   & \ttt{D*\_1+}   &  10533 & $B_{1s}^0$   & \ttt{B\_1s0}   \\
 20423 & $D^{*0}_1$   & \ttt{D*\_10}   &  10541 & $B^{*+}_{0c}$& \ttt{B*\_0c+}  \\
 20433 & $D^{*+}_{1s}$& \ttt{D*\_1s+}  &  10543 & $B_{1c}^+$   & \ttt{B\_1c+}   \\
 20443 & $\chi_{1c}$  & \ttt{chi\_1c}  &  10551 & $\chi_0^b$   & \ttt{chi\_0b}  \\
100443 & $\psi'$      & \ttt{psi'}     &  10553 & $h_{1b}$     & \ttt{h\_1b}    \\
       &              &                &  20513 & $B^{*0}_1$   & \ttt{B*\_10}   \\
       &              &                &  20523 & $B^{*+}_1$   & \ttt{B*\_1+}   \\
       &              &                &  20533 & $B^{*0}_{1s}$& \ttt{B*\_1s0}  \\
       &              &                &  20543 & $B^{*+}_{1c}$& \ttt{B*\_1c+}  \\
       &              &                &  20553 & $\chi_{1b}$  & \ttt{chi\_1b}  \\
       &              &                & 100553 & $\Upsilon'$  & \ttt{Upsilon'} \\
\hline
\end{tabular}
\end{center}
\end{table}


\begin{table}[ptb]
\captive{Heavy baryon codes\protect\label{t:codesix} }
\vspace{1ex}
\begin{center}
\begin{tabular}{|c|c|c||c|c|c||c|c|c|@{\protect\rule{0mm}{\tablinsep}}} \hline
KF & Name & Printed & KF & Name & Printed & KF & Name & Printed \\ \hline
 4112 & $\Sigma_c^0$          & \ttt{Sigma\_c0}     & 5112 & $\Sigma_b^-$     & \ttt{Sigma\_b-}  &
 5432 & $\Omega^{'0}_{bc}$    & \ttt{Omega'\_bc0}   \\
 4114 & $\Sigma*_c^0$         & \ttt{Sigma*\_c0}    & 5114 & $\Sigma^{*-}_b$  & \ttt{Sigma*\_b-} &
 5434 & $\Omega^{*0}_{bc}$    & \ttt{Omega*\_bc0}   \\
 4122 & $\Lambda_c^+$         & \ttt{Lambda\_c+}    &  5122 & $\Lambda_b^0$   & \ttt{Lambda\_b0} &
 5442 & $\Omega_{bcc}^+$      & \ttt{Omega\_bcc+}   \\
 4132 & $\Xi_c^0$             & \ttt{Xi\_c0}        &  5132 & $\Xi_b^-$       & \ttt{Xi\_b-}     &
 5444 & $\Omega^{*+}_{bcc}$   & \ttt{Omega*\_bcc+}  \\
 4212 & $\Sigma_c^+$          & \ttt{Sigma\_c+}     &  5142 & $\Xi_{bc}^0$    & \ttt{Xi\_bc0}    &
 5512 & $\Xi_{bb}^-$          & \ttt{Xi\_bb-}       \\
 4214 & $\Sigma*_c^+$         & \ttt{Sigma*\_c+}    &  5212 & $\Sigma_b^0$    & \ttt{Sigma\_b0}  &
 5514 & $\Xi^{*-}_{bb}$       & \ttt{Xi*\_bb-}      \\
 4222 & $\Sigma_c^{++}$       & \ttt{Sigma\_c++}    &  5214 & $\Sigma^{*0}_b$ & \ttt{Sigma*\_b0} &
 5522 & $\Xi_{bb}^0$          & \ttt{Xi\_bb0}       \\
 4224 & $\Sigma*_c^{++}$      & \ttt{Sigma*\_c++}   &  5222 & $\Sigma_b^+$    & \ttt{Sigma\_b+}  &
 5524 & $\Xi^{*0}_{bb}$       & \ttt{Xi*\_bb0}      \\
 4232 & $\Xi_c^+$             & \ttt{Xi\_c+}        &  5224 & $\Sigma^{*+}_b$ & \ttt{Sigma*\_b+} &
 5532 & $\Omega_{bb}^-$       & \ttt{Omega\_bb-}    \\
 4312 & $\Xi^{'0}_c$          & \ttt{Xi'\_c0}       &  5232 & $\Xi_b^0$       & \ttt{Xi\_b0}     &
 5534 & $\Omega^{*-}_{bb}$    & \ttt{Omega*\_bb-}   \\
 4314 & $\Xi^{*0}_c$          & \ttt{Xi*\_c0}       &  5242 & $\Xi_{bc}^+$    & \ttt{Xi\_bc+}    &
 5542 & $\Omega_{bbc}^0$      & \ttt{Omega\_bbc0}   \\
 4322 & $\Xi^{'+}_c$          & \ttt{Xi'\_c+}       &  5312 & $\Xi^{'-}_b$    & \ttt{Xi'\_b-}    &
 5544 & $\Omega^{*0}_{bbc}$   & \ttt{Omega*\_bbc0}  \\
 4324 & $\Xi^{*+}_c$          & \ttt{Xi*\_c+}       &  5314 & $\Xi^{*-}_b$    & \ttt{Xi*\_b-}    &
 5554 & $\Omega^{*-}_{bbb}$   & \ttt{Omega*\_bbb-}  \\
 4332 & $\Omega_c^0$          & \ttt{Omega\_c0}     &  5322 & $\Xi^{'0}_b$    & \ttt{Xi'\_b0}    &
      &                       &                     \\
 4334 & $\Omega^{*0}_c$       & \ttt{Omega*\_c0}    &  5324 & $\Xi^{*0}_b$    & \ttt{Xi*\_b0}    &
      &                       &                     \\
 4412 & $\Xi_{cc}^+$          & \ttt{Xi\_cc+}       &  5332 & $\Omega_b^-$    & \ttt{Omega\_b-}  &
      &                       &                     \\
 4414 & $\Xi^{*+}_{cc}$       & \ttt{Xi*\_cc+}      &  5334 & $\Omega^{*-}_b$ & \ttt{Omega*\_b-} &
      &                       &                     \\
 4422 & $\Xi_{cc}^{++}$       & \ttt{Xi\_cc++}      &  5342 & $\Omega_{bc}^0$ & \ttt{Omega\_bc0} &
      &                       &                     \\
 4424 & $\Xi_{cc}^{*++}$      & \ttt{Xi*\_cc++}     &  5412 & $\Xi^{'0}_{bc}$ & \ttt{Xi'\_bc0}   &
      &                       &                     \\
 4432 & $\Omega_{cc}^+$       & \ttt{Omega\_cc+}    &  5414 & $\Xi^{*0}_{bc}$ & \ttt{Xi*\_bc0}   &
      &                       &                     \\
 4434 & $\Omega^{*+}_{cc}$    & \ttt{Omega*\_cc+}   &  5422 & $\Xi^{'+}_{bc}$ & \ttt{Xi'\_bc+}   &
      &                       &                     \\
 4444 & $\Omega_{ccc}^{*++}$  & \ttt{Omega*\_ccc++} &  5424 & $\Xi^{*+}_{bc}$ & \ttt{Xi*\_bc+}   &
      &                       &                     \\
\hline
\end{tabular}
\end{center}
\end{table}

\subsection*{B: Particle family codes}\label{app:code}

JAM has an other particle code which is rough identification of
 the particles comparison to the PDG code.
It can be obtained from \ttt{kchg(kc,5)}.

\begin{verbatim}
c...Elementary particles.
      common/partdat3/id_quark,id_lept,id_exc,id_boson,id_diq
     $       ,id_tec,id_susy,id_special
c...Mesons.
      common/partdat2/id_pi,id_light1,id_light0
     $    ,id_str,id_charm,id_bott,id_cc,id_bb,id_mdiff
c...Baryons.
      common/partdat1/id_nucl,id_nucls,id_delt,id_delts
     $     ,id_lamb,id_lambs,id_sigm,id_sigms,id_xi,id_xis,id_omega
     $    ,id_charmb,id_bottb,id_bdiff
\end{verbatim}

Actual values are stored in \ttt{block data jamdata}.



\begin{thebibliography}{20}

\bibitem{fritiof}
      B. Andersson, G. Gustafson and H. Pi, \Journal{\ZPC}{57}{485}{1993};\\
      H. Pi, \Journal{\CPC}{71}{173}{1992}.

\bibitem{luciae}
     Sa Ben-Hao and Tai An, \Journal{\CPC}{90}{121}{1995};
     \Journal{\PRC}{55}{2010}{1997};
     \Journal{\PLB}{399}{29}{1997};
     \Journal{\PLB}{409}{393}{1997}.

\bibitem{dpm}
        A. Capella, U. Sukhatme, C.-I. Tan and J. Tran Thanh Van,
        \Journal{\PR}{236}{225}{1994}.

\bibitem{venus} K. Werner,
   \Journal{\ZPC}{42}{85}{1989};
   \Journal{\PR}{232}{87}{1993}.\\
 VENUS, 
{\em http://www-subatech.in2p3.fr/Sciences/Theorie/venus/venus.html}.

\bibitem{pythia} T. Sj{\" o}strand, \Journal{\CPC}{82}{74}{1994};
      {\em http://www.thep.lu.se/tf2/staff/torbjorn/Pythia.html}.

\bibitem{hijing}
        X. N. Wang and M. Gyulassy, \Journal{\PRD}{44}{3501}{1991};
        X. N. Wang, \Journal{\PR}{280}{287}{1997};
        X. N. Wang and M. Gyulassy, \Journal{\CPC}{83}{307}{1994};
        {\em http://www-nsdth.lbl.gov/\verb+~+xnwang/hijing/}.


\bibitem{hijet} A. Shor, R. S. Longacre,
	\Journal{\PLB}{218}{100}{1989}.


\bibitem{rqmd1}
H. Sorge, H. St\"ocker and W. Greiner, \Journal{\ANP}{192}{266}{1989};
H. Sorge, A. von Keitz, R. Mattiello, H. St\"ocker and W. Greiner,
  \Journal{\ZPC}{47}{629}{1990}.
H. Sorge, L. Winckelmann, H. St\"ocker and  W. Greiner,
 \Journal{\ZPC}{59}{85}{1993}.

\bibitem{rqmd2}
H. Sorge, \Journal{\PRC}{52}{3291}{1995}.


\bibitem{qgsm}
L. Bravina, L.P. Csernai, P. Levai, and D. Dtrottman,
  \Journal{\PRC}{51}{2161}{1994}.

\bibitem{arc}
Y. Pang, T. J. Schlagel, S. H. Kahana,
   \Journal{\NPA}{544}{435c}{1992};
   \Journal{\PRL}{68}{2743}{1992};
 S. H. Kahana, D. H. Kahana, Y. Pang, and T. J. Schlagel,
  \Journal{\ARPS}{46}{31}{1996}.
 

\bibitem{art}
  B. A. Li and C. M. Ko,  \Journal{\PRC}{52}{2037}{1995}.
\Journal{\PLB}{382}{337}{1996};
\Journal{\NPA}{630}{556}{1998};
\Journal{\PRC}{58}{R1382}{1998};
B. A. Li, C.M. Ko, and G.Q. Li, \Journal{\PRC}{54}{844}{1996}.

\bibitem{hsd}
 W. Ehehalt and W. Cassing \Journal{\NPA}{602}{449}{1996};
 J. Geiss, W. Cassing and C. Greiner, nucl-th/9805012;
 W. Cassing and E. L. Bratkovskaya, \Journal{\PR}{308}{65}{1999}.

\bibitem{urqmd}
L.A. Winckelmann, S.A. Bass, M. Bleicher, M. Brandstetter, A. Dumitru,
C. Ernst, L. Gerland, J. Konopka, S. Soff, C. Spieles, H. Weber,
C. Hartnack, J. Aichelin, N. Amelin, H. St\"ocker and W. Greiner,
  Nucl. Phys. {\bf A610}, 116c (1996); nucl-th/9610033;
%
%
S.A. Bass, M. Belkacem, M. Bleicher, M. Brandstetter, L. Bravina, C. Ernst,
L. Gerland, M. Hofmann, S. Hofmann, J. Konopka, G. Mao, L. Neise, S. Soff,
C. Spieles, H. Weber, L.A. Winckelmann, H. St\"ocker, W.  Greiner,
C. Hartnack, J. Aichelin and  N. Amelin,
\Journal{\PPNP}{41}{225}{1998}; nucl-th/9803035;
%
%
M Bleicher, E. Zabrodin, C. Spieles, S. A. Bass, C. ernst,
S. Soff, L. Bravina, M. Belkacem, H. Weber, H. St\"ocker and W. Greiner,
\Journal{\JP}{G25}{1859}{1999}.






\bibitem{buu}
        G. F. Bertsch and S. Das Gupta, 
        \Journal{\PR}{160}{189}{1988}.

\bibitem{inc1}
K. Chen,  Z. Frankel, G.Friedlander, J. R. Grover, J. M. Miller
and Y. Shimamoto, \Journal{\PR}{166}{1968}{949}.
\bibitem{inc2}
Y. Yariv and Z. Frankel, \Journal{\PRC}{20}{2227}{1979}.
\bibitem{inc3}
J. Cugnon, \Journal{\PRC}{22}{1885}{1980};
J. Cugnon, T. Mizutani and J. Vandermeulen,
          \Journal{\NPA}{352}{505}{1981}.

\bibitem{rbuu}
	W.Cassing and U.Mosel,
	\Journal{\PPNP}{25}{235}{1990}.

\bibitem{qmd}
        J. Aichelin, 
        \Journal{\PR}{202}{233}{1991}.
\bibitem{vni}
K. Geiger and B. M{\"u}ller, \Journal{\NPB}{369}{600}{1992};
K. Geiger, VNI,
        \Journal{\PR}{258}{238}{1995};
        \Journal{\CPC}{104}{70}{1997};
 {\em http://rhic.phys.columbia.edu/rhic/vni}.

\bibitem{zpc}
       B. Zhang, ZPC, \Journal{\CPC}{109}{70}{1997};
      {\em http://nt1.phys.columbia.edu/people/bzhang/ZPC/zpc.html}.

\bibitem{cascade1} Y. Pang, GCP,
    {\em http://rhic.phys.columbia.edu/rhic/gcp.}.

\bibitem{bin1}
	B. Zhang and Y. Pang,
        \Journal{\PRC}{56}{2185}{1997}.
\bibitem{bin2}
	B. Zhang, M. Gyulassy  and Y. Pang,
         \Journal{\PRC}{58}{1175}{1998}.

\bibitem{flow}
R. Mattiello, H. Sorge,  H. St\"ocker and W. Greiner,
  \Journal{\PRC}{55}{1443}{1997}.

\bibitem{eskola1} K. J. Eskola and X. N. Wang, \Journal{\PRD}{49}{1282}{1994}.


\bibitem{jam}
Y. Nara, N. Otuka, A. Ohnishi, K. Niita, S.  Chiba,
   \Journal{\PRC}{61}{024901}{2000},   e-Print: nucl-th/9904059;
Y. Nara, \Journal{\NPA}{638}{555c}{1998},
 {\em http://quark.phy.bnl.gov/\verb+~+ynara/jam/}.


\bibitem{CernHera}
{\it Total Cross-Sections for Reactions of High Energy Particles}
  vol. 12a and vol. 12b
edited by A. Baldni, V. Flaminio, W. G. Moorhead and D. R. O. Morrison
(Springer-Verlag Berlin 1988).

\bibitem{Teis}
 S. Teis, W. Cassing, M. Effenberger, A. Hombach, U. Mosel and Gy. Wolf,
  \Journal{\ZPA}{356}{421}{1997}.


% detbal
\bibitem{Wolf2}
        Gy.Wolf, W.Cassing, U.Mosel,
        \Journal{\NPA}{552}{549}{1993}.
\bibitem{detbal1}
        P. Danielewicz and G. F. Bertch,
        \Journal{\NPA}{533}{712}{1991}.
\bibitem{detbal2}
        Bao-An Li, \Journal{\NPA}{552}{605}{1993}.

\bibitem{Kitazoe}
	Y. Kitazoe,
	\Journal{\PTP}{73}{1191}{1985}.

\bibitem{bialas}
	A. Bialas, M. Gyulassy,
	\Journal{\NPB}{291}{793}{1987}.
\bibitem{dwid1} J. Randrup, \Journal{\NPA}{314}{429}{1979)};
          Rittenberg, \Journal{\RMP}{43}{S1}{1971}.

\bibitem{PDG96}
Particle-Data-Group, \newblock Phys. Rev. {\bf D54} (1996).

\bibitem{ref:slope} See for example,
   \Journal{\PTPsuppl}{41,42}{p291}{1967}.

\bibitem{goulianos} K. Goulianos, \Journal{\PR}{101}{1983}{169}.


\bibitem{Brown}
        G. E. Brown, C. M. Ko, Z. G. Wu and L. H. Xia,
        \Journal{\PRC}{43}{1881}{1991}.

\bibitem{lund}
  B. Andersson, G. Gustafson, G. Ingelman and T. Sj\"ostrand,
     \Journal{\PR}{97}{31}{1983}.
\bibitem{lund2}
  T. Sj\"ostrand,
     \Journal{\NP}{B248}{469}{1984}.



\bibitem{niita}
K. Niita, S. Chiba, T. Maruyama, T. Maruyama, H. Takada, T. Fukahori,
 Y. Nakahara and A. Iwamoto,
    \Journal{\PRC}{52}{2620}{1995}.

\bibitem{KK}
   Y. Nara, A. Ohnishi, T. Harada and A. Engel,
    \Journal{\NPA}{614}{433}{1997}.

\bibitem{mdraw}
   Toshiki Maruyama, the graphic software {\it mdraw} can be obtained
   from \\
{\em http://hadron31.tokai.jaeri.go.jp/msource/Draw/draw.html}.
\bibitem{angel}
   Koji Niita, the graphic software {\it angel} can be obtained
   from the request on e-mail: {\em niita@hadron03.tokai.jaeri.go.jp}.


\end{thebibliography}{}



\end{document}

